\documentclass[parskip=full,11pt,twoside]{scrartcl}
\usepackage[utf8]{inputenc}

\title{VIPER Interactive Prolog Education Runtime}
\subtitle{Entwurf}
\author{Paul Brinkmeier, Lukas Brocke, Jannik Koch, Aaron Maier, Christian Oder}

% section numbers in margins:
\renewcommand\sectionlinesformat[4]{\makebox[0pt][r]{#3}#4}

% header & footer
\usepackage{scrlayer-scrpage}
\lofoot{\today}
\refoot{\today}
\pagestyle{scrheadings}

\usepackage{amsmath} % for $\text{}$

\usepackage[sfdefault,light]{roboto}
\usepackage[T1]{fontenc}
\usepackage[german]{babel}
\usepackage[yyyymmdd]{datetime} % must be after babel
\renewcommand{\dateseparator}{-} % ISO8601 date format
\usepackage{hyperref}
\usepackage[nameinlink]{cleveref}
\crefname{figure}{Abb}{Abb}
\usepackage[section]{placeins}
\usepackage{xcolor}
\usepackage{graphicx}
\usepackage{listings}
\usepackage{courier}
\usepackage{enumitem}
\hypersetup{
	pdftitle={Entwurf},
}

\usepackage{csquotes}

\newcommand\urlpart[2]{$\underbrace{\text{\texttt{#1}}}_{\text{#2}}$}

\lstset{basicstyle=\ttfamily,breaklines=true}

% Don't strech across whole page
\raggedbottom

% Start new page with each section
\usepackage{sectsty}
\sectionfont{\clearpage}

\begin{document}
\maketitle

\section{Einleitung}

\section{Klassenbeschreibungen}

\subsection{Paket \texttt{edu.kit.ipd.pp.viper.view}}

\subsubsection{Klasse \texttt{Button}}

\textbf{\texttt{public Button(String textKey, Command command)}}
\setlist{nolistsep}
\begin{itemize}[noitemsep]
	\item[-] \texttt{textKey}: Schlüssel für die Übersetzung des angezeigten Texts
	\item[-] \texttt{command}: Auszuführender Befehl beim Betätigen der Schaltfläche
\end{itemize}
Initialisiert eine neue Schaltfläche für die GUI. Beim Betätigen der Schaltfläche wird der übergebene Befehl \texttt{command} ausgeführt.

\subsubsection{Klasse \texttt{ConsoleInputField}}

\textbf{\texttt{public ConsoleInputField()}}\\
Initialisiert das Eingabefeld für Prolog-Abfragen.

\textbf{\texttt{public String getText()}}
\setlist{nolistsep}
\begin{itemize}[noitemsep]
	\item[-] Returns \texttt{String}: Inhalt des Eingabefeldes
\end{itemize}
Gibt den aktuellen Inhalt des Eingabefeldes zurück.

\textbf{\texttt{public void clear()}}\\
Leert das Eingabefeld.

\textbf{\texttt{public void lock()}}\\
Sperrt das Eingabefeld, sodass keine Eingaben durch den Nutzer möglich sind.

\textbf{\texttt{public void unlock()}}\\
Gibt das Eingabefeld frei, sodass Eingaben durch den Nutzer möglich sind. Das Eingabefeld wird vor der Freigabe geleert.

\subsubsection{Klasse \texttt{ConsoleOutputArea}}

\textbf{\texttt{public ConsoleOutputArea()}}\\
Initialisiert die Konsole, welche für textuelle Ausgaben vom Interpreter sowie für Fehlermeldungen verwendet wird.

\textbf{\texttt{public void printLine(String line, Color color)}}
\setlist{nolistsep}
\begin{itemize}[noitemsep]
	\item[-] \texttt{line}: Der anzuzeigende Text
	\item[-] \texttt{color}: Gewünschte Farbe für den gesamten Text
\end{itemize}
Gibt eine neue Zeile mit Text \texttt{line} in der Farbe \texttt{color} auf der Konsole aus.

\textbf{\texttt{public void clear()}}\\
Leert den Inhalt der Konsole.

\subsubsection{Klasse \texttt{ConsolePanel}}

\textbf{\texttt{public ConsolePanel()}}\\
Initialisiert den GUI-Bereich für die Konsole, sowie das Eingabefeld.

\textbf{\texttt{public void clearAll()}}\\
Leert sowohl die Konsole, als auch das Eingabefeld.

\textbf{\texttt{public void clearOutputArea()}}\\
Leert den Inhalt der Konsole.

\textbf{\texttt{public void clearInputField()}}\\
Leert das Eingabefeld.

\textbf{\texttt{public void lockInput()}}\\
Deaktiviert das Eingabefeld, sodass keine Eingaben mehr möglich sind.

\textbf{\texttt{public void unlockInput()}}\\
Aktiviert das Eingabefeld, sodass Eingben möglich sind. Der Inhalt wird bei der Aktivierung geleert.

\subsubsection{Klasse \texttt{EditorPanel}}

\textbf{\texttt{public EditorPanel()}}\\
Initialisiert den GUI-Bereich für den Editor.

\textbf{\texttt{public String getSourceText()}}\\
Gibt den aktuellen Inhalt des Editors zurück.

\textbf{\texttt{public void setSourceText(String text)}}
\setlist{nolistsep}
\begin{itemize}[noitemsep]
	\item[-] \texttt{text}: Gewünschter Text
\end{itemize}
Setzt den Inhalt des Editors. Der vorherige Inhalt wird dabei verworfen.

\subsubsection{Klasse \texttt{LanguageResources}}

\textbf{\texttt{private LanguageManager()}}\\
Initialisiert die Klasse zur Verwaltung der Übersetzungen.

\textbf{\texttt{public static LanguageManager getInstance()}}\\
Gibt eine Instanz des \texttt{LanguageManager}s zurück, realisiert also das Einzelstück-Entwurfsmuster.

\textbf{\texttt{public void setLocale(Locale locale)}}
\setlist{nolistsep}
\begin{itemize}[noitemsep]
	\item[-] \texttt{locale}: Gewünschte Sprache
\end{itemize}
Setzt die gewünschte Sprache auf \texttt{locale}, diese Einstellung wird außerdem im Dateisystem des Betriebssystems gespeichert. Ein Aufruf dieser Methode hat die Aktualisierung der gesamten GUI zur Folge.\\
Ist die gewünschte Sprache nicht unterstützt, so wird eine \texttt{UnsupportedLocaleException} geworfen.

\textbf{\texttt{public static String getString(String key)}}
\setlist{nolistsep}
\begin{itemize}[noitemsep]
	\item[-] \texttt{key}: Sprach-Schlüssel für angeforderte Übersetzung.
	\item[-] Returns \texttt{String}: Übersetzung für aktuell gesetzte Sprache.
\end{itemize}
Gibt die Übersetzung für den Schlüssel \texttt{key} in der aktuell gesetzten Sprache zurück.

\subsubsection{Klasse \texttt{MainWindow}}

\textbf{\texttt{public MainWindow()}}\\
Initialisiert das Hauptfenster mit allen seinen Bereichen (Panels) und der Menüleiste.

\subsubsection{Klasse \texttt{MenuBar}}

\textbf{\texttt{public MenuBar()}}\\
Initialisiert die Menüleiste mit allen verfügbaren Einträgen (\texttt{MenuItem}s).

\subsubsection{Klasse \texttt{MenuItem}}

\textbf{\texttt{public MenuItem(String textKey, Command command)}}\\
Initialisiert ein neues Element in der Menüleiste.

\subsubsection{Klasse \texttt{Observable}}

\textbf{\texttt{public Observable()}}\\
Erstellt eine neue Instanz. Da die Klasse \texttt{Observable} abstrakt ist, ist dies nur durch eine Unterklasse möglich.

\textbf{\texttt{public void addListener(Observer listener)}}
\setlist{nolistsep}
\begin{itemize}[noitemsep]
	\item[-] \texttt{listener}: Neuer Beobachter, der benachrichtigt werden kann.
\end{itemize}
Fügt einen neuen Beobachter hinzu, der später mit \texttt{notifyAll} benachrichtigt werden kann.

\textbf{\texttt{public void notifyAll()}}\\
Benachrichtigt alle vorher durch \texttt{addListener()} hinzugefügte Beobachter.

\subsubsection{Interface \texttt{Observer}}

\textbf{\texttt{public void update()}}\\
Aktualisiert die Komponente nach einem Sprachwechsel, dies bedeutet i.d.R. die neue richtige Übersetzung über den \texttt{LanguageManager} zu laden.

\subsubsection{Klasse \texttt{ToolBar}}

\textbf{\texttt{public ToolBar()}}\\
Initialisiert die Werkzeugleiste des Hauptfensters.

\subsubsection{Klasse \texttt{VisualisationPanel}}

\textbf{\texttt{public VisualisationPanel()}}\\
Initialisiert den GUI Bereich für die Visualisierung.

\subsubsection{Klasse \texttt{VisualisationViewer}}

\textbf{\texttt{public VisualisationViewer()}}\\
Initialisiert einen Betrachter für SVGs, der den generierten Graphen anzeigen kann. Tasten- und Mausbewegungen für die Navigation werden ebenso unterstützt.

\textbf{\texttt{public void updateGraph(String graph)}}\\
Initialisiert einen Betrachter für SVGs, der den generierten Graphen anzeigen kann. Tasten- und Mausbewegungen für die Navigation werden ebenso unterstützt.

\section{Ablaufdiagramme}

\appendix

\section{Anhang}

\subsection{Klassendiagramm}

\end{document}
