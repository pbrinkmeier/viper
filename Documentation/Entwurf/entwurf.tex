\documentclass[parskip=full,11pt,twoside]{scrartcl}
\usepackage[utf8]{inputenc}

\title{VIPER Interactive Prolog Education Runtime}
\subtitle{Entwurf}
\author{Paul Brinkmeier, Lukas Brocke, Jannik Koch, Aaron Maier, Christian Oder}

% section numbers in margins:
\renewcommand\sectionlinesformat[4]{\makebox[0pt][r]{#3}#4}

% header & footer
\usepackage{scrlayer-scrpage}
\lofoot{\today}
\refoot{\today}
\pagestyle{scrheadings}

\usepackage{amsmath} % for $\text{}$

\usepackage[sfdefault,light]{roboto}
\usepackage[T1]{fontenc}
\usepackage[german]{babel}
\usepackage[yyyymmdd]{datetime} % must be after babel
\renewcommand{\dateseparator}{-} % ISO8601 date format
\usepackage{hyperref}
\usepackage[nameinlink]{cleveref}
\crefname{figure}{Abb}{Abb}
\usepackage[section]{placeins}
\usepackage{xcolor}
\usepackage{graphicx}
\usepackage{listings}
\usepackage{courier}
\hypersetup{
	pdftitle={Entwurf},
}

\usepackage{csquotes}

\newcommand\urlpart[2]{$\underbrace{\text{\texttt{#1}}}_{\text{#2}}$}

\lstset{basicstyle=\ttfamily,breaklines=true}

% Don't strech across whole page
\raggedbottom

% Start new page with each section
\usepackage{sectsty}
\sectionfont{\clearpage}

\begin{document}
\maketitle

\section{Einleitung}

VIPER wird nach dem \enquote{Model-View-Controller}-Architekturstil entworfen.

Die \enquote{Model}-Komponente kapselt interne Datenstrukturen, welche wiederum von der \enquote{View}-Komponente (in diesem Fall einer GUI für Desktop-Anwendungen) graphisch dargestellt werden. Dazu gehören u.a. der Inhalt des Editors oder des Konsolen-Ausgabefensters. Die View-Komponente ist mit Java und der GUI-Bibliothek Swing implementiert. Die Model-Komponente funktioniert unabhängig von der View-Komponente.

Interaktionen mit der Software durch die View-Komponente werden über die \enquote{Controller}-Komponente verarbeitet. Dadurch entstehende Änderungen, bspw. eine Eingabe in den Editor, nehmen wiederum Einfluss auf die internen Datenstrukturen in der Model-Komponente und damit auf die View-Komponente.

Die Pakete orientieren sich an den vom Architekturstil festgelegten Unterteilungen. Dementsprechend wird zwischen drei Paketen unterschieden. Das \enquote{edu.kit.ipd.pp.viper.model}-Paket beinhaltet die Logik des Programms. Das \enquote{edu.kit.ipd.pp.viper.view}-Paket beinhaltet die komplette GUI und das \enquote{edu.kit.ipd.pp.viper.controller}-Paket beinhaltet Schnittstellen für die Kommunikation zwischen den anderen beiden Paketen.

Weiter existieren die Pakete \enquote{edu.kit.ipd.pp.viper.model.parser} und \enquote{edu.kit.ipd.pp.viper.model.interpreter} für die Implementierung des Prolog-Parsers und des Prolog-Interpreters. Diese befinden sich im Model-Paket, da sie lediglich Algorithmen zur Ver- und Abarbeitung von Prolog-Programmen und Abfragen beinhalten.
Außerdem existiert das Paket \enquote{edu.kit.ipd.pp.viper.view.visualization} zur Implementierung der Visualisierung. Dieses befindet sich im View-Paket, da es einen Teil der GUI beinhaltet. 

\section{Klassenbeschreibungen}

\section{Ablaufdiagramme}

\appendix

\section{Anhang}

\subsection{Klassendiagramm}

\end{document}
