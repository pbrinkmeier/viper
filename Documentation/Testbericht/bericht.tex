\documentclass[parskip=full,11pt,twoside]{scrartcl}
\usepackage[utf8]{inputenc}

\title{VIPER Interactive Prolog Education Runtime}
\subtitle{Testbericht}
\author{Paul Brinkmeier, Lukas Brocke, Jannik Koch, Aaron Maier, Christian Oder}
\date{}

% section numbers in margins:
\renewcommand\sectionlinesformat[4]{\makebox[0pt][r]{#3}#4}

% header & footer
\usepackage{scrlayer-scrpage}
\lofoot{\today}
\refoot{\today}
\pagestyle{scrheadings}

\usepackage{amsmath} % for $\text{}$

\usepackage[sfdefault,light]{roboto}
\usepackage[T1]{fontenc}
\usepackage[german]{babel}
\usepackage[yyyymmdd]{datetime} % must be after babel
\renewcommand{\dateseparator}{-} % ISO8601 date format
\usepackage{hyperref}
\usepackage[nameinlink]{cleveref}
\crefname{figure}{Abb}{Abb}
\usepackage[section]{placeins}
\usepackage{xcolor}
\usepackage{graphicx}
\usepackage{listings}
\usepackage{courier}
\usepackage{enumitem}
\usepackage{dirtree}
\usepackage{pgfplots}
\usepackage{pgfgantt}
\usepackage{pifont}
\usepackage{multicol}
\hypersetup{
	pdftitle={Testbericht},
}

\usepackage{csquotes}

\newcommand\urlpart[2]{$\underbrace{\text{\texttt{#1}}}_{\text{#2}}$}
\newcommand{\cmark}{\ding{51}}
\newcommand{\xmark}{\ding{55}}

\lstset{basicstyle=\ttfamily,breaklines=true}

% Don't strech across whole page
\raggedbottom

% Start new page with each section
\usepackage{sectsty}
\sectionfont{\clearpage}

\begin{document}
\pagenumbering{roman}
\maketitle

\tableofcontents

\section{Einleitung}
\pagenumbering{arabic}
\setcounter{page}{1}

In der Testphase wird VIPER durch weitere Unittests sowie die im Pflichtenheft vorgestellten Testfall-Szenarien getestet. Außerdem wird das Programm in Hinblick auf Nutzbarkeit und Codequalität geprüft und überarbeitet.

Die Unittests werden mit dem Framework JUnit \footnote{\url{https://junit.org/}} geschrieben. Die durch diese Tests erreichte Abdeckung wird mit JaCoCo \footnote{\url{https://www.eclemma.org/jacoco/}} überprüft. Das Ziel ist eine Abdeckung nahe 100\% für das Controller Paket sowie alle Model Pakete. Mittels JAssertSwing \footnote{\url{https://joel-costigliola.github.io/assertj/assertj-swing.html}} wird auch das View Paket von VIPER so weit wie möglich automatisch getestet.

Für einheitlich formattieren Quelltext wird Checkstyle \footnote{\url{https://checkstyle.org/}} verwendet. Die Konfiguration stammt von der Vorlesung Programmieren und wird um einige striktere Regeln erweitert. Die Werkzeuge FindBugs \footnote{\url{http://findbugs.sourceforge.net/}} und PMD \footnote{\url{https://pmd.github.io/}} werden eingesetzt, um den Quelltext zur Übersetzungszeit auf Fehler und schlechten Programmierstil zu überprüfen.

Um VIPER auf Nutzbarkeit zu prüfen, wird das Programm von außenstehenden Personen mit und ohne Vorkenntnisse in Prolog verwendet. Dabei entstehende Meinungen und beobachtete Reaktionen gehen in die Überarbeitung des Programms mit ein.

Für die Verwaltung von gefundenen Fehlern und zu implementierenden Features werden die Issues des GitLab Repositories verwendet. Diese erlauben u.a. Vorlagen, Gewichtungen sowie das Zuweisen von zuständigen Personen. Außerdem werden die genannten Werkzeuge in das Gradle Buildsystem integriert, sodass die im GitLab konfigurierte Continuous Integration das Projekt testen und bauen kann.

\section{Tests}

\subsection{Unit-Tests}

% Alle Tests auflisten, die existieren, mit nem simplen Symbol markieren, welche seit der QA neu dazugekommen sind

% Wie GUI getestet (JAssertSwing)?

\subsection{Testfälle aus Pflichtenheft}

\subsection{Statische Analyse und Linting}

\subsubsection{FindBugs}

Die statische Analyse des ByteCode mit FindBugs brachte die folgenden Ergebnisse hervor:

\begin{itemize}
  \item \textbf{Fehlende \texttt{hashCode} Methoden}\\
        In den Klassen des Models wurde für notwendige Vergleiche die Methode \texttt{java.lang.Object\#equals(Object)} überschrieben, die Methode \texttt{java.lang.Object\#hashCode()} allerdings nicht. Diese wurde in allen betroffenen Klassen ergänzt.
  \item \textbf{Schreiben in Klassenattribut}\\
        Die Hauptklasse (\texttt{MainWindow}) schreibt bei Initialisierung einen Wahrheitswert in das statische Klassenattribut \texttt{debug}, abhängig davon ob VIPER im Debug-Modus gestartet wurde oder nicht. Da die Abfrage dieses Wahrheitswertes allerdings statisch, d.h. ohne Instanz der Hauptklasse, möglich sein soll, wird diese Warnung ignoriert.
  \item \textbf{Rückgabewert \texttt{null} statt \texttt{Optional}}\\
        In der Methode \texttt{ActivationRecord\#getPrevious()} wurde in der letzten Zeile \texttt{null} statt einem \texttt{Optional<ActivationRecord>} zurückgegeben. Diese Zeile konnte nie erreicht werden, wurde aber vom Compiler gefordert. Die Methode wurde als Reaktion auf die Warnung von FindBugs umstrukturiert und neu implementiert.
\end{itemize}

\subsubsection{PMD}

Das Werkzeug PMD wurde sowohl zur Überprüfung von Semantik, als auch möglicher Fehler im Programm genutzt. Ebenfalls wurde der Quelltext auf unnötige und unerreichbare Zeilen sowie suboptimalen Code überprüft.

Das Ergebnis der Analyse sorgte für die folgenden Änderungen:

\begin{itemize}
  \item \textbf{Variablennamen überarbeitet}\\
        Sehr lange und sehr kurze Variablennamen wurden zugunsten der Verständlichkeit des Quelltextes umbenannt.
  \item \textbf{Aufruf von \texttt{super()}}\\
        Konstruktoren von Unterklassen, so beispielsweise alle Commands, wurden für guten Stil um einen Aufruf von \texttt{super()} ergänzt.
  \item \textbf{Finale Klassenattribute}\\
        Bereits in Konstruktoren initialisierte Klassenattribute wurden um das Schlüsselwort \texttt{final} ergänzt, um eine erneute, unerwünschte Zuweisung unmöglich zu machen.
  \item \textbf{Fest-codierte Konstanten in statische Klassenattribute ausgelagert}\\
        Im Quelltext verwendete Konstanten, wie beispielsweise von Swing zurückgegebene Zahlen bei betätigten Schaltflächen eines Dialogfensters, wurden in finale, statische Klassenattribute ausgelagert.
  \item \textbf{\texttt{return} als letztes Statement}\\
        Sofern möglich wurden alle Methoden so umgeschrieben, dass die letzte Zeile des Methodenkörpers ein \texttt{return} Statement darstellt. Dies verbessert die Lesbarkeit des Quelltextes durch weniger frühzeitige Methodenabbrüche.
  \item \textbf{Vermeiden von String-Konkatenation}\\
        Aufgrund der Unveränderbarkeit (\enquote{immutability}) von Strings in Java wurde im Quelltext verwendete String-Konkatenation für bessere Performance durch die \texttt{StringBuilder} Klasse ersetzt.
\end{itemize}

\subsubsection{Checkstyle}

Für einen einheitlichen Codestil im gesamten Projekt wird der Quelltext inklusive aller Tests von Checkstyle untersucht. Die Konfiguration stammt von der Vorlesung Programmieren \footnote{\url{https://sdqweb.ipd.kit.edu/wiki/Vorlesung_Programmieren_WS16/17}} und umfasst u.a. Regeln für maximale Klassengröße, Parameterzahl für Methoden, Leerzeichen vor und hinter geschwungenen Klammern, Einrückung mit Leerzeichen und einige mehr.

Die folgenden, eigenen Regeln erweitern die Konfiguration speziell für VIPER:

\begin{itemize}
  \item Geschweifte Klammern bei \texttt{if ... else} Blöcken und Schleifen sind immer notwendig, auch wenn der Körper nur eine Zeile umfasst.
  \item Das Schlüsselwort \texttt{this} muss immer verwendet werden, wenn auf Klassen-Attribute und -Methoden zugegriffen wird.
  \item Die maximale Parameteranzahl für Methoden beträgt 10, da einige Konstruktoren in Controller-Klassen viele Referenzen auf GUI Elemente oder andere Commands brauchen.
\end{itemize}

\subsection{Nutzerstudien}

Um VIPER von Außenstehenden testen und beurteilen zu lassen, wurden im Rahmen der Qualitätssicherungs-Phase mehrere Nutzerstudien durchgeführt.

Die Teilnehmer besaßen dabei unterschiedlich viele Vorkenntnisse in Prolog. Teilnehmern ohne Vorkenntnisse wurde die Funktionsweise anhand des \texttt{simpsons.pl} Beispielprogramms, welches mit VIPER mitgeliefert wird, kurz erläutert.

In den Studien lag der Fokus auf den folgenden Fragestellungen:

\begin{itemize}
  \item Wie intuitiv ist der Programm-Ablauf (Code eingeben, Parsen, Abfrage stellen, Änderungen machen, wieder Parsen, ...)?
  \item Lässt sich die Visualisierung gut bedienen? Wie funktioniert Zoomen und Bewegen? Sind die Funktionen der verschiedenen Schritte-Schaltflächen ersichtlich?
  \item Ist die grafische Oberfläche sinnvoll angeordnet? Findet man alle Funktionalitäten ohne lange suchen zu müssen?
  \item Ist die Menüleiste sinnvoll strukturiert? Erschließt sich die Funktion jeder Schaltfläche in der Werkzeugleiste durch das Icon?
  \item Genereller Eindruck vom Programm: Wie stabil läuft es, wie schnell reagiert das Programm Eingaben und Schaltflächen, wie performant sind Abfragen?
\end{itemize}

% Ergebnisse des Usability Testings (Reaktionen der Leute!)
% Icons geändert/Buttons aufgeteilt, Zoomen, Shortcuts, Stdlib anzeigen
% Zoom-Debatte mit Meinung von Max

\subsubsection{Studie 1} % Max

\subsubsection{Studie 2} % Alex

\subsubsection{Studie 3} % Sebi

\subsubsection{Studie 4} % Lennart

\section{Gefundene Fehler}

\begin{enumerate}[label=\#\arabic*]
  \item \textbf{\enquote{Nächster Schritt} Schaltfläche wird fälschlicherweise deaktiviert}\\
        Wird während einer Interpretation die \enquote{Vorheriger Schritt} Schaltfläche und anschließend die \enquote{Nächster Schritt} Schaltfläche betätigt, so wird danach letztere Schaltfläche deaktiviert und kann nicht mehr betätigt werden, obwohl es noch Schritte geben kann.\\\\
        \textbf{Grund:} Im Controller für die \enquote{Vorheriger Schritt} Schaltfläche wird der falsche Status der Schaltflächen (\texttt{ClickableState}) gesetzt. Dadurch werden die Schaltflächen nach einmaligem Betätigen der \enquote{Nächster Schritt} Schaltfläche deaktiviert.\\
        \textbf{Behebung:} Im entsprechenden Controller muss der korrekte Status der Schaltflächen gesetzt werden.

  \item \textbf{\enquote{Abbrechen} Schaltfläche wird nicht aktiviert}\\
        Wird die \enquote{Nächste Lösung} Schaltfläche betätigt und die Interpretation dauert etwas länger, so wird die \enquote{Abbrechen} Schaltfläche nicht aktiviert. Die Interpretation kann also nicht abgebrochen werden.\\\\
        \textbf{Grund:} Der Fall, dass vor dem Betätigen der \enquote{Nächste Lösung} kein Einzelschritt gemacht wird, wurde nicht berücksichtigt. In diesem Fall wird der falsche Status der Schaltflächen (\texttt{ClickableState}) gesetzt.\\
        \textbf{Behebung:} Der entsprechende Controller muss den korrekten \texttt{ClickableState} setzen.

  \item \textbf{Standardbibliothek kann nicht korrekt geladen werden}\\
        Wird VIPER als eine kompilierte \texttt{.jar} Datei gestartet, so wirft das Parsen eines Prolog-Programms bei aktivierter Standardbibliothek eine \texttt{FileSystemNotFoundException}. Das Parsen ist trotzdem möglich, es erscheint eine Fehlermeldung in der Konsole und die Regeln aus der Standardbibliothek sind nicht verfügbar.\\\\
        \textbf{Grund:} Während der Entwicklung wird VIPER über das verwendete Build-Management-Tool Gradle gestartet. Die Exception tritt in diesem Fall nicht auf.\\
        \textbf{Behebung:} Die in der \texttt{.jar} Datei enthaltene Standardbibliothek muss über einen Java Stream eingelesen werden, anstatt als Datei behandelt und geöffnet zu werden.

  \item \textbf{Im \enquote{Standardbibliothek anzeigen} Fenster ist kein Scrollen möglich}\\
        Über den Menüpunkt \enquote{Standardbibliothek anzeigen} lässt sich der Inhalt der Standardbibliothek in einem neuen Fenstern anzeigen. Bei langem Quelltext ist in diesem Fenster kein Scrollen möglich, der Inhalt wird am unteren Fensterrand abgeschnitten.\\\\
        \textbf{Grund:} Bevor die Standardbibliothek erweitert wurde, war der Quelltext nie zu lang für das Fenster. Somit ist der Fehler nie aufgefallen.\\
        \textbf{Behebung:} Das zum Anzeigen des Inhalts verwendete Textfeld muss in einem Java Swing \texttt{JScrollPane} eingebettet werden, damit Scrollen möglich ist.
\end{enumerate}

\section{Zusätzliche Features}

Im Zuge der Verbesserung der Gebrauchstauglichkeit (\enquote{usability}) von VIPER wurde das Programm um einige Funktionalitäten erweitert.\\
Diese haben sich u.a. aus den Rückmeldungen von Testpersonen unseres Programms ergeben.\\\\
Im Folgenden werden die zusätzlich implementieren Funktionalitäten und Änderungen kurz erläutert.

\begin{itemize}
  \item \textbf{Nächste Lösung}\\
        Die Funktion der \enquote{Nächste Lösung} Schaltfläche wurde dahingehend verändert, dass nun die nächste auftretende Lösung, ausgehend vom aktuell angezeigten Schritt, ausgegeben wird. Bisher wurde bis zur nächsten, noch nicht berechneten Lösung gesprungen.
  \item \textbf{Letzte Lösung}\\
        Um für eine eingegebene Abfrage alle möglichen Lösungen anzeigen zu lassen, wurde die \enquote{Letzte Lösung} Schaltfläche in der Werkzeugleiste eingeführt. Diese berechnet direkt alle Schritte der Abfrage und zeigt alle möglichen Lösungen in der Konsole an.
  \item \textbf{Indikator für erfolgreiche Unifikation}\\
        Ist die Unifikation eines (Teil-)Ziels erfolgreich, so wird der dazugehörige Knoten in der Visualisierung nun grün eingefärbt und um ein Häkchen (\ding{51}) ergänzt.
  \item \textbf{Menüpunkt \enquote{Hilfe}}\\
        Die Menüleiste wurde um den Eintrag \enquote{Hilfe} erweitert, der Optionen zum Anzeigen einer Anleitung für das Programm, des Inhalts der Standardbibliothek sowie eines \enquote{Über}-Dialogs enthält.
  \item \textbf{Anleitung}\\
        In der neu hinzugekommenen Anleitung wird dem Nutzer die grundlegende Arbeitsweise des Programms nahegelegt. Zusätzlich enthält sie eine Legende zu den Bedeutungen der Farben in der Konsole, sowie eine Hilfe zur Bedienung des Editors, der Visualisierung und der Schritt-Schaltflächen.
  \item \textbf{Inhalt der Standardbibliothek}\\
        Die Standardbibliothek wurde um einige, aus der Vorlesung \enquote{Programmierparadigmen} bekannte, Regeln ergänzt. Dazu gehören beispielsweise Listen-Sortierung, Quadratwurzel-Berechnung und Summenbildung.
  \item \textbf{Hinweis bei Namenskonflikten mit der Standardbibliothek}\\
        Falls im geparsten Programm Regeln vorkommen, die Namenskonflikte mit Regeln aus der Standardbibliothek hervorrufen, so wird eine Warnung in der Konsole ausgegeben. Ist die Standardbibliothek nicht aktiviert, so wird der Hinweis nicht angezeigt.
  \item \textbf{Beispielprogramme}\\
        Mehrere Beispiel-Programme, darunter ein Bekanntes mit den Regeln \texttt{father/2}, \texttt{mother/2} und \texttt{grandparent/2}, werden nun mit VIPER mitgeliefert und können über den \enquote{Datei}-Menüpunkt in den Editor geladen werden.
  \item \textbf{Tastenkombinationen für den Editor}\\
        Um die Nutzbarkeit des Editors zu verbessern, wurde er um gebräuchliche Tastaturkürzel wie Strg+O zum Öffnen und Strg+S zum Speichern einer Datei erweitert.
  \item \textbf{Mitgelieferte Schriftart für Visualisierung}\\
        Um ein einheitliches Aussehen der Visualisierung auf allen Betriebssystemen garantieren zu können, wird die freie Schriftart \enquote{Liberation Serif} von Red Hat im Programm mitgeliefert. Diese wird für alle Texte und Zeichen in der Visualisierung verwendet.
  \item \textbf{Zoom an der Position des Mauszeigers}\\
        Wird in der Visualisierung mittels des Mausrads der angezeigte Graph vergrößert oder verkleinert, so wird die Richtung des Zooms über die Position des Mauszeigers über dem Graphen bestimmt. Bisher wurde in der Visualisierung zentral herein- bzw. herausgezoomt. Außerdem wurde die Sensitivität beim Zoomen mit dem Mausrad verringert.
  \item \textbf{Änderung der Schriftgröße in Editor und Konsole}\\
        Um die Schriftgröße des Editors und der Konsole anpassen zu können, wurden zwei Schaltflächen \enquote{+} und \enquote{-} ähnlich denen in der Visualisierung hinzugefügt. Die Schaltflächen befinden sich in der Werkzeugleiste und verändern gleichzeitig die Schriftgröße des Editors sowie der Konsole. Diese Einstellung wird außerdem gespeichert und bleibt nach einem Programm-Neustart erhalten.
  \item \textbf{Platzhalter-Hinweise}\\
        Wurde noch keine Visualisierung gestartet, so wird im Visualisierungs-Bereich der Hinweis \enquote{Bitte Quelltext parsen und eine Abfrage eingeben, um die Visualisierung zu starten} angezeigt. Außerdem ist im Eingabefeld der Hinweis \enquote{Neue Abfrage eingeben...} sichtbar, sollte es nicht fokussiert sein.
  \item \textbf{Darstellung des Mauszeigers}\\
        Der Mauszeiger ändert seine Darstellung zur \enquote{Laden...}-Animation, wenn die Berechnung der nächsten Lösung etwas Zeit in Anspruch nimmt.
\end{itemize}

\section{Statistiken}

% Testabdeckung mit JUnit, möglichst detailliert (Tests pro Paket, LoC an Tests pro Palet, wie viele Tests laufen vollautomatisch).
% Commits und LoC der Phase insg.
% Monkey Testing Ergebnisse

\end{document}
