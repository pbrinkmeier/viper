\documentclass[parskip=full,11pt,twoside]{scrartcl}
\usepackage[utf8]{inputenc}

\title{VIPER Interactive Prolog Education Runtime}
\subtitle{Testbericht}
\author{Paul Brinkmeier, Lukas Brocke, Jannik Koch, Aaron Maier, Christian Oder}
\date{}

% section numbers in margins:
\renewcommand\sectionlinesformat[4]{\makebox[0pt][r]{#3}#4}

% header & footer
\usepackage{scrlayer-scrpage}
\lofoot{\today}
\refoot{\today}
\pagestyle{scrheadings}

\usepackage{amsmath} % for $\text{}$

\usepackage[sfdefault,light]{roboto}
\usepackage[T1]{fontenc}
\usepackage[german]{babel}
\usepackage[yyyymmdd]{datetime} % must be after babel
\renewcommand{\dateseparator}{-} % ISO8601 date format
\usepackage{hyperref}
\usepackage[nameinlink]{cleveref}
\crefname{figure}{Abb}{Abb}
\usepackage[section]{placeins}
\usepackage{xcolor}
\usepackage{graphicx}
\usepackage{listings}
\usepackage{courier}
\usepackage{enumitem}
\usepackage{dirtree}
\usepackage{pgfplots}
\usepackage{pgfgantt}
\usepackage{pifont}
\usepackage{multicol}
\hypersetup{
	pdftitle={Testbericht},
}

\usepackage{csquotes}

\newcommand\urlpart[2]{$\underbrace{\text{\texttt{#1}}}_{\text{#2}}$}
\newcommand{\cmark}{\ding{51}}
\newcommand{\xmark}{\ding{55}}

\lstset{basicstyle=\ttfamily,breaklines=true}

% Don't strech across whole page
\raggedbottom

% Start new page with each section
\usepackage{sectsty}
\sectionfont{\clearpage}

\begin{document}
\pagenumbering{roman}
\maketitle

\tableofcontents

\section{Einleitung}
\pagenumbering{arabic}
\setcounter{page}{1}

% Anschluss an Implementierungsbericht, erreichte Abdeckung etc. Ein Satz pro Phase nach Vorbild der bisherigen Dokumente. Danach kurz etwas zur aktuellen Phase

% Arbeitsweise (GitLab Issues, Assignees, Gewichtungen, ...)

% Verwendete Werkzeuge und Verfahren (statische Tools, JUnit, JAssertSwing, andere Leute vor unser Programm gesetzt)

\section{Tests}

\subsection{Unit-Tests}

% Alle Tests auflisten, die existieren, mit nem simplen Symbol markieren, welche seit der QA neu dazugekommen sind

% Wie GUI getestet (JAssertSwing)?

\subsection{Testfälle aus Pflichtenheft}

\subsection{Integrationstests}

\subsection{Statische Analyse}

% Checkstyle, Linter, sonstige Tools für Codequalität

% Ergebnisse FindBugs, PMD (Zahlen, welche Fehlergruppe wir ignorieren, z.B. hashCode(), if..else curly braces)

\section{Gefundene Fehler}

\begin{enumerate}[label=\#\arabic*]
  \item \textbf{\enquote{Nächster Schritt} Schaltfläche wird fälschlicherweise deaktiviert}\\
        Wird während einer Interpretation die \enquote{Vorheriger Schritt} Schaltfläche verwendet und anschließend die \enquote{Nächster Schritt} Schaltfläche betätigt, so wird danach letztere Schaltfläche deaktiviert.\\\\
        \textbf{Symptome:} Die \enquote{Nächster Schritt} Schaltfläche wird deaktiviert und kann nicht mehr betätigt werden, obwohl es noch Schritte geben kann.\\
        \textbf{Behebung:} Im entsprechenden Controller muss der korrekte Status der Schaltflächen (\texttt{ClickableState}) gesetzt werden.

  \item \textbf{\enquote{Abbrechen} Schaltfläche wird nicht aktiviert}\\
        Wird die \enquote{Nächste Lösung} Schaltfläche betätigt und die Interpretation dauert etwas länger, so wird die \enquote{Abbrechen} Schaltfläche nicht aktiviert. Die Interpretation kann also nicht abgebrochen werden.\\\\
        \textbf{Symptome:} Die \enquote{Abbrechen} Schaltfläche bleibt deaktiviert.\\
        \textbf{Behebung:} Der entsprechende Controller muss den korrekten Status der Schaltflächen (\texttt{ClickableState}) setzen, wenn ein Thread für die Berechnung der nächsten Lösung gestartet wird.

  \item \textbf{Standardbibliothek kann nicht korrekt geladen werden}\\
        Wird VIPER als eine kompilierte \texttt{.jar} Datei gestartet, so wirft das Parsen eines Prolog-Programms bei aktivierter Standardbibliothek eine \texttt{FileSystemNotFoundException}. Das Parsen ist trotzdem möglich, es erscheint eine Fehlermeldung in der Konsole und die Regeln aus der Standardbibliothek sind nicht verfügbar.\\\\
        \textbf{Symptome:} In der Konsole erscheint eine Fehlermeldung. Wurde die \texttt{.jar} Datei über eine Konsole gestartet, so wird dort ein Stacktrace der Java Virtual Machine (JVM) ausgegeben.\\
        \textbf{Behebung:} Die in der \texttt{.jar} Datei enthaltene Standardbibliothek muss über einen Java Stream eingelesen werden, anstatt als Datei behandelt und geöffnet zu werden.

  \item \textbf{Im \enquote{Standardbibliothek anzeigen} Fenster ist kein Scrollen möglich}\\
        Über den Menüpunkt \enquote{Standardbibliothek anzeigen} lässt sich der Inhalt der Standardbibliothek in einem neuen Fenstern anzeigen. Bei langem Code ist in diesem Fenster kein Scrollen möglich, der Inhalt wird am unteren Fensterrand abgeschnitten.\\\\
        \textbf{Symptome:} Im geöffneten \enquote{Standardbibliothek anzeigen} Fenster ist Code, der über den unteren Fensterrand hinausgeht, nicht mehr sichtbar.\\
        \textbf{Behebung:} Das zum Anzeigen des Inhalts verwendete Textfeld muss in einem Java Swing \texttt{JScrollPane} eingebettet werden, damit Scrollen möglich ist.
\end{enumerate}

\section{Zusätzliche Features}

Im Zuge der Verbesserung der Gebrauchstauglichkeit (\enquote{usability}) von VIPER wurde das Programm um einige Funktionalitäten erweitert.\\
Diese haben sich u.a. aus den Rückmeldungen von Testpersonen unseres Programms ergeben.\\\\
Im Folgenden werden die zusätzlich implementieren Funktionalitäten und Änderungen kurz erläutert.

\begin{itemize}
  \item \textbf{Nächste Lösung}\\
        Die Funktion der \enquote{Nächste Lösung} Schaltfläche wurde dahingehend verändert, dass nun die nächste auftretende Lösung, ausgehend vom aktuell angezeigten Schritt, ausgegeben wird. Bisher wurde bis zur nächsten, noch nicht berechneten Lösung gesprungen.
  \item \textbf{Letzte Lösung}\\
        Um für eine eingegebene Abfrage alle möglichen Lösungen anzeigen zu lassen, wurde die \enquote{Letzte Lösung} Schaltfläche in der Werkzeugleiste eingeführt. Diese berechnet direkt alle Schritte der Abfrage und zeigt alle möglichen Lösungen in der Konsole an.
  \item \textbf{Indikator für erfolgreiche Unifikation}\\
        Ist die Unifikation eines (Teil-)Ziels erfolgreich, so wird der dazugehörige Knoten in der Visualisierung nun grün eingefärbt und um ein Häkchen (\ding{51}) ergänzt.
  \item \textbf{Visualisierung des Cut}\\
        \textit{todo}
  \item \textbf{Menüpunkt \enquote{Hilfe}}\\
        Die Menüleiste wurde um den Eintrag \enquote{Hilfe} erweitert, der Optionen zum Anzeigen einer Anleitung für das Programm, des Inhalts der Standardbibliothek sowie eines \enquote{Über}-Dialogs enthält.
  \item \textbf{Anleitung}\\
        In der neu hinzugekommenen Anleitung wird dem Nutzer die grundlegende Arbeitsweise des Programms nahegelegt. Zusätzlich enthält sie eine Legende zu den Bedeutungen der Farben in der Konsole, sowie eine Hilfe zur Bedienung des Editors, der Visualisierung und der Schritt-Schaltflächen.
  \item \textbf{Inhalt der Standardbibliothek}\\
        Die Standardbibliothek wurde um einige, aus der Vorlesung \enquote{Programmierparadigmen} bekannte, Regeln ergänzt. Dazu gehören beispielsweise Listen-Sortierung, Quadratwurzel-Berechnung und Summenbildung.
  \item \textbf{Hinweis bei Namenskonflikten mit der Standardbibliothek}\\
        Falls im geparsten Programm Regeln vorkommen, die Namenskonflikte mit Regeln aus der Standardbibliothek hervorrufen, so wird eine Warnung in der Konsole ausgegeben. Ist die Standardbibliothek nicht aktiviert, so wird der Hinweis nicht angezeigt.
  \item \textbf{Beispielprogramme}\\
        Mehrere Beispiel-Programme, darunter ein Bekanntes mit den Regeln \texttt{father/2}, \texttt{mother/2} und \texttt{grandparent/2}, werden nun mit VIPER mitgeliefert und können über den \enquote{Datei}-Menüpunkt in den Editor geladen werden.
  \item \textbf{Tastenkombinationen für den Editor}\\
        Um die Nutzbarkeit des Editors zu verbessern, wurde er um gebräuchliche Tastaturkürzel wie Strg+O zum Öffnen und Strg+S zum Speichern einer Datei erweitert.
  \item \textbf{Zoom an der Position des Mauszeigers}\\
        Wird in der Visualisierung mittels des Mausrads der angezeigte Graph vergrößert oder verkleinert, so wird die Richtung des Zooms über die Position des Mauszeigers über dem Graphen bestimmt. Bisher wurde in der Visualisierung zentral herein- bzw. herausgezoomt. Außerdem wurde die Sensitivität beim Zoomen mit dem Mausrad verringert.
  \item \textbf{Änderung der Schriftgröße in Editor und Konsole}\\
        Um die Schriftgröße des Editors und der Konsole anpassen zu können, wurden zwei Schaltflächen \enquote{+} und \enquote{-} ähnlich denen in der Visualisierung hinzugefügt. Die Schaltflächen befinden sich in der Werkzeugleiste und verändern gleichzeitig die Schriftgröße des Editors sowie der Konsole. Diese Einstellung wird außerdem gespeichert und bleibt nach einem Programm-Neustart erhalten.
  \item \textbf{Platzhalter-Hinweise}\\
        Wurde noch keine Visualisierung gestartet, so wird im Visualisierungs-Bereich der Hinweis \enquote{Bitte Quellcode parsen und eine Abfrage eingeben, um die Visualisierung zu starten} angezeigt. Außerdem ist im Eingabefeld der Hinweis \enquote{Neue Abfrage eingeben...} sichtbar, sollte es nicht fokussiert sein.
  \item \textbf{Darstellung des Mauszeigers}\\
        Der Mauszeiger ändert seine Darstellung zur \enquote{Laden...}-Animation, wenn die Berechnung der nächsten Lösung etwas Zeit in Anspruch nimmt.
\end{itemize}

\section{Nutzertests}

% Ergebnisse des Usability Testings (Reaktionen der Leute!)
% Icons geändert/Buttons aufgeteilt, Zoomen, Shortcuts, Stdlib anzeigen
% Zoom-Debatte mit Meinung von Max

\section{Statistiken}

% Testabdeckung mit JUnit, möglichst detailliert (Tests pro Paket, LoC an Tests pro Palet, wie viele Tests laufen vollautomatisch).
% Commits und LoC der Phase insg.
% Monkey Testing Ergebnisse

\end{document}
