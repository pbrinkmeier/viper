\documentclass[parskip=full,11pt,twoside]{scrartcl}
\usepackage[utf8]{inputenc}

\title{VIPER Interactive Prolog Education Runtime}
\subtitle{Testbericht}
\author{Paul Brinkmeier, Lukas Brocke, Jannik Koch, Aaron Maier, Christian Oder}

% section numbers in margins:
\renewcommand\sectionlinesformat[4]{\makebox[0pt][r]{#3}#4}

% header & footer
\usepackage{scrlayer-scrpage}
\lofoot{\today}
\refoot{\today}
\pagestyle{scrheadings}

\usepackage{amsmath} % for $\text{}$

\usepackage[sfdefault,light]{roboto}
\usepackage[T1]{fontenc}
\usepackage[german]{babel}
\usepackage[yyyymmdd]{datetime} % must be after babel
\renewcommand{\dateseparator}{-} % ISO8601 date format
\usepackage{hyperref}
\usepackage[nameinlink]{cleveref}
\crefname{figure}{Abb}{Abb}
\usepackage[section]{placeins}
\usepackage{xcolor}
\usepackage{graphicx}
\usepackage{listings}
\usepackage{courier}
\usepackage{enumitem}
\usepackage{dirtree}
\usepackage{pgfplots}
\usepackage{pgfgantt}
\usepackage{pifont}
\usepackage{multicol}
\hypersetup{
	pdftitle={Testbericht},
}

\usepackage{csquotes}

\newcommand\urlpart[2]{$\underbrace{\text{\texttt{#1}}}_{\text{#2}}$}
\newcommand{\cmark}{\ding{51}}%
\newcommand{\xmark}{\ding{55}}%

\lstset{basicstyle=\ttfamily,breaklines=true}

% Don't strech across whole page
\raggedbottom

% Start new page with each section
\usepackage{sectsty}
\sectionfont{\clearpage}

\begin{document}
\pagenumbering{roman}
\maketitle
\tableofcontents

\section{Einleitung}
\pagenumbering{arabic}
\setcounter{page}{1}

% Anschluss an Implementierungsbericht, erreichte Abdeckung etc.

% Arbeitsweise (GitLab Issues, Assignees, Gewichtungen, ...)

% Verwendete Werkzeuge und Verfahren (statische Tools, JUnit, JAssertSwing, andere Leute vor unser Programm gesetzt)

\section{Tests}

\subsection{Unit-Tests}

% Grobe Liste, welche Dinge wir getestet haben

% Wie GUI getestet (JAssertSwing)?

\subsection{Integrationstests}

% Tests aus Pflichtenheft

\subsection{Statische Analyse}

% Checkstyle, Linter, sonstige Tools für Codequalität

\section{Gefundene Fehler}

\begin{enumerate}[label=\#\arabic*]
  \item \textbf{Hier steht die kurze Beschreibung des Fehlers}\\
        Hier steht eine ausführlichere Beschreibung inklusive der \enquote{Symptome} (was sieht man?). Warum tritt der Fehler auf?\\\\
        \textbf{Behebung:} Was genau wurde geändert um es zu beheben?
\end{enumerate}

\section{Zusätzliche Features und Änderungen}

Im Zuge der Verbesserung der Gebrauchstauglichkeit (\enquote{usability}) von VIPER wurde das Programm um einige Funktionalitäten erweitert.\\
Diese haben sich sowohl durch eigene Benutzung, als auch durch .. anderer Personen ergeben.\\\\
Im Folgenden werden die zusätzlich implementieren Funktionalitäten und Änderungen kurz erläutert.

\begin{itemize}
  \item \textbf{Nächste Lösung}\\
        Die Funktion der \enquote{Nächste Lösung} Schaltfläche wurde dahingehend verändert, dass nun die nächste auftretende Lösung, ausgehend vom aktuell angezeigten Schritt, ausgegeben wird. Bisher wurde bis zur nächsten, noch nicht berechneten Lösung gesprungen.
  \item \textbf{Letzte Lösung}\\
        Um für eine eingegebene Abfrage alle möglichen Lösungen anzeigen zu lassen, wurde die \enquote{Letzte Lösung} Schaltfläche in der Werkzeugleiste eingeführt. Diese berechnet direkt alle Schritte der Abfrage und zeigt alle möglichen Lösungen in der Konsole an.
  \item \textbf{Indikator für erfolgreiche Unifikation}\\
        Ist die Unifikation eines (Teil-)Ziels erfolgreich, so wird der dazugehörige Knoten in der Visualisierung nun grün eingefärbt und um ein Häkchen (\ding{51}) ergänzt.
  \item \textbf{Visualisierung des Cut}\\
        \textit{todo}
  \item \textbf{Menu "Über"}\\
        \textit{todo}
  \item \textbf{Anleitung}\\
        \textit{todo}
  \item \textbf{Inhalt Stdlib}\\
        \textit{todo}
  \item \textbf{Hinweis Stdlib überschreiben}\\
        \textit{todo}
  \item \textbf{Beispielprogramme}\\
        \textit{todo}
  \item \textbf{Tastenkombinationen für den Editor}\\
        Um die Nutzbarkeit ... wurde der Editor um gebräuchliche Tastaturkürzel, namentlich ... und ... erweitert.
  \item \textbf{Zoom wo Maus ist}\\
        \textit{todo}
  \item \textbf{Text Zoom, Größe speichern}\\
        \textit{todo}
  \item \textbf{Placeholder Text und Graph}\\
        \textit{todo}
  \item \textbf{Maus Spinner}\\
        \textit{todo}

\end{itemize}
% Ergebnisse des Usability Testings (Reaktionen der Leute!)
% Icons geändert/Buttons aufgeteilt, Zoomen, Shortcuts, Stdlib anzeigen

\section{Statistiken}

% Testabdeckung mit JUnit
% Wie viele andere Leute haben das Programm getestet
% Monkey Testing Ergebnisse

\end{document}
