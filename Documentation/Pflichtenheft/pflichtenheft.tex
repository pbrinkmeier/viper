\documentclass[parskip=full,11pt,twoside]{scrartcl}
\usepackage[utf8]{inputenc}

\title{VIPER: Viper Interactive Prolog Education Runtime}
\author{Paul Brinkmeier, Luke Brocke, Christian Oder, Aaron Maier, Jannik Koch}

% section numbers in margins:
\renewcommand\sectionlinesformat[4]{\makebox[0pt][r]{#3}#4}

% header & footer
\usepackage{scrlayer-scrpage}
\lofoot{\today}
\refoot{\today}
\pagestyle{scrheadings}

\usepackage{amsmath} % for $\text{}$

\usepackage[sfdefault,light]{roboto}
\usepackage[T1]{fontenc}
\usepackage[german]{babel}
\usepackage[yyyymmdd]{datetime} % must be after babel
\renewcommand{\dateseparator}{-} % ISO8601 date format
\usepackage{hyperref}
\usepackage[nameinlink]{cleveref}
\crefname{figure}{Abb}{Abb}
\usepackage[section]{placeins}
\usepackage{xcolor}
\usepackage{graphicx}
\hypersetup{
	pdftitle={Pflichtenheft},
	bookmarks=true,
}
\usepackage{csquotes}

\newcommand\urlpart[2]{$\underbrace{\text{\texttt{#1}}}_{\text{#2}}$}

\usepackage{pflichtenheft}

\begin{document}
\maketitle

\section{Einleitung}

Einleitende Worte.

\pagebreak
\section{Kriterien}
% Diese Section sollte kurz und knapp "für Manager" sein
% und auf eine Seite passen.

\subsection{Muss}

\criterium{Interpretation eines Prolog-Programms mit begrenztem Umfang}{crt:interpretation}

Das Programm kann eine Teilmenge der Prologsprache interpretieren.

\subsection{Kann}

\criteriumOptional{Export von Visualisierungsbäumen als Bilddatei}{crt:export}

Das Programm kann einen Visualisierungsbaum als Bilddatei im PNG-Format exportieren.

\subsection{Abgrenzung}

\criteriumNot{Voller Sprachsupport}{crt:fullsupport}

Die Software hat kein Verständnis für Prolog-Sprachfeatures außerhalb der vorgestellten Prolog-Teilmenge.

\criteriumNot{Andere Programmiersprachen}{crt:otherlanguages}

Die Entwicklungsumgebung beschränkt sich auf die Prolog-Sprache und bietet keine offiziell Unterstützung für Prolog-Dialekte oder andere Programmiersprachen jeglicher Art.

\criteriumNot{Nutzung in einer Kommandozeile}{crt:cli}

Das Produkt beschränkt sich auf eine graphische Darstellung über GUI-Elemente. Eine Interaktion über die Konsole ist nicht unterstützt.


\pagebreak
%%%%%%%%%%%%%%
\section{Produkteinsatz}

Das Produkt soll als graphisches Prolog-Lerntool betrieben werden.

Die Zielgruppe des Lerntools sind Studierende und Enthusiasten.

Das Produkt soll sich auf eine Teilmenge der Prolog-Sprache beschränken, diese jedoch voll unterstützen.

Das Produkt soll für die Teilmenge der Sprache eine integrierte Entwicklungsumgebung darstellen.

Neben einer Entwicklungsumgebung für die unterstützte Teilmenge von Prolog stehen Visualisierungs-Features zum Verständnis der Sprachumsetzung zur Verfügung.

\section{Produktumgebung}

Das Programm soll als graphische Applikation auf einem Desktop-System betrieben werden.

Es stehen mindestens 2 AMD64/x86 Kerne mit insgesamt 2GB shared RAM zur Verfügung.

Unterstützte Betriebssysteme sind Windows ab Version 7 und aufwärts, Mac OSX 10.9 aufwärts sowie Ubuntu Linux 16.04.

Eine Maus sowie eine Tastatur sind als Eingabegeräte angeschlossen und funktionsfähig.

Eine Installation des Java Runtime Environments Version 8 aufwärts ist auf dem System vorhanden.

%%%%%%%%%%%
\section{Funktionale Anforderungen}

\functionality{Interpreter}{fnc:interpreter}
\fulfills{crt:interpretation}

Um Prolog zu interpretieren wird ein selbstgeschriebener Interpreter genutzt.

%%%%%%%%%%%
\section{Nicht-Funktionale Anforderungen}

\nonFunctionality{Modernes Design}{nfc:design}

Das Design soll modern und seriös wirken.

%%%%%%%%%%%
\section{Tests}

\test{Interpretierung von Hello World}{tst:helloworld}
\tests{fnc:interpreter}

\teststep{Nutzer \enquote{Max Mustermann} hat ein Standard Hello-World Programm in den Editor eingegeben.}
{Max klickt auf die Run-Schaltfläche des Hauptfensters.}
{Das Programm gibt die Ausgabe \enquote{Hello World} zurück.}

%%%%%%%%%%%%%
\pagebreak
\appendix

\section{Seitenentwürfe}

\begin{figure}[hb]
\fbox{\includegraphics[width=\textwidth]{image/example.png}}
\caption{\label{fig:editor}
Editor Hauptfenster
}
\end{figure}

\section{Glossar}

\textbf{Nutzer}:
Eine Person, welche den Dienst nutzt.

\end{document}
