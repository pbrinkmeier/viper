\documentclass[parskip=full,11pt,twoside]{scrartcl}
\usepackage[utf8]{inputenc}

\title{VIPER Interactive Prolog Education Runtime}
\subtitle{Implementierungsbericht}
\author{Paul Brinkmeier, Lukas Brocke, Jannik Koch, Aaron Maier, Christian Oder}

% section numbers in margins:
\renewcommand\sectionlinesformat[4]{\makebox[0pt][r]{#3}#4}

% header & footer
\usepackage{scrlayer-scrpage}
\lofoot{\today}
\refoot{\today}
\pagestyle{scrheadings}

\usepackage{amsmath} % for $\text{}$

\usepackage[sfdefault,light]{roboto}
\usepackage[T1]{fontenc}
\usepackage[german]{babel}
\usepackage[yyyymmdd]{datetime} % must be after babel
\renewcommand{\dateseparator}{-} % ISO8601 date format
\usepackage{hyperref}
\usepackage[nameinlink]{cleveref}
\crefname{figure}{Abb}{Abb}
\usepackage[section]{placeins}
\usepackage{xcolor}
\usepackage{graphicx}
\usepackage{listings}
\usepackage{courier}
\usepackage{enumitem}
\usepackage{dirtree}
\usepackage{multicol}
\hypersetup{
	pdftitle={Implementierungsbericht},
}

\usepackage{csquotes}

\newcommand\urlpart[2]{$\underbrace{\text{\texttt{#1}}}_{\text{#2}}$}

\lstset{basicstyle=\ttfamily,breaklines=true}

% Don't strech across whole page
\raggedbottom

% Start new page with each section
\usepackage{sectsty}
\sectionfont{\clearpage}

\begin{document}
\pagenumbering{roman}
\maketitle
\tableofcontents

\section{Einleitung}
\pagenumbering{arabic}
\setcounter{page}{1}

% Anschluss an Entwurfsdokument
Bei VIPER handelt es sich um ein grafisches Tool zur Unterstützung des Lernens und Lehrens der logischen Programmiersprache Prolog. Hierbei soll der Nutzer die Interpretation von Prolog-Abfragen durch eine Visualisierung in Form einer Baumstruktur nachvollziehen können. Weiter wird ein Editor zur Erstellung von Prolog-Quelltext sowie eine Konsole zur Eingabe von Abfragen und Ausgabe von Informationen für den Nutzer bereitgestellt.

Die Entwicklung von VIPER erfolgt in der Programmiersprache Java anhand des Model-View-Controller-Architekturstils. Hiernach wird die grafische Darstellungs-Komponente (View) von der internen Logik (Model) getrennt. Die Vermittlung zwischen diesen Komponenten erfolgt über den Controller. Die grafische View-Komponente ist hierbei für Desktop-Nutzung ausgelegt und mit Swing implementiert. Die Generierung der Visualisierung erfolgt mit Graphviz über die Graphviz-Java Implementation, das Darstellen der erzeugten Visualisierung nutzt die Apache Batik Bibliothek.

\section{Änderungen zum Entwurf und Pflichtenheft}

\section{Fortschritt der Implementierung}

% Muss-Kriterien alle drin?
% Welche Kann-Kriterien?

\section{Implementierungsplan}

% Abweichungen?

\section{Tests und Abdeckung}

% Testabdeckung

\section{Statistiken}

% Commits, LoC, JavaDoc, ...

\end{document}
