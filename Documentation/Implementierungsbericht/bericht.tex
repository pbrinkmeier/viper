\documentclass[parskip=full,11pt,twoside]{scrartcl}
\usepackage[utf8]{inputenc}

\title{VIPER Interactive Prolog Education Runtime}
\subtitle{Implementierungsbericht}
\author{Paul Brinkmeier, Lukas Brocke, Jannik Koch, Aaron Maier, Christian Oder}

% section numbers in margins:
\renewcommand\sectionlinesformat[4]{\makebox[0pt][r]{#3}#4}

% header & footer
\usepackage{scrlayer-scrpage}
\lofoot{\today}
\refoot{\today}
\pagestyle{scrheadings}

\usepackage{amsmath} % for $\text{}$

\usepackage[sfdefault,light]{roboto}
\usepackage[T1]{fontenc}
\usepackage[german]{babel}
\usepackage[yyyymmdd]{datetime} % must be after babel
\renewcommand{\dateseparator}{-} % ISO8601 date format
\usepackage{hyperref}
\usepackage[nameinlink]{cleveref}
\crefname{figure}{Abb}{Abb}
\usepackage[section]{placeins}
\usepackage{xcolor}
\usepackage{graphicx}
\usepackage{listings}
\usepackage{courier}
\usepackage{enumitem}
\usepackage{dirtree}
\usepackage{pgfplots}
\usepackage{pgfgantt}
\usepackage{pifont}
\usepackage{multicol}
\hypersetup{
	pdftitle={Implementierungsbericht},
}

\usepackage{csquotes}

\newcommand\urlpart[2]{$\underbrace{\text{\texttt{#1}}}_{\text{#2}}$}
\newcommand{\cmark}{\ding{51}}%
\newcommand{\xmark}{\ding{55}}%

\lstset{basicstyle=\ttfamily,breaklines=true}

% Don't strech across whole page
\raggedbottom

% Start new page with each section
\usepackage{sectsty}
\sectionfont{\clearpage}

\begin{document}
\pagenumbering{roman}
\maketitle
\tableofcontents

\section{Einleitung}
\pagenumbering{arabic}
\setcounter{page}{1}

% Anschluss an Entwurfsdokument
Bei VIPER handelt es sich um ein grafisches Tool zur Unterstützung des Lernens und Lehrens der logischen Programmiersprache Prolog. Hierbei soll der Nutzer die Interpretation von Prolog-Abfragen nachvollziehen können. Dies wird durch die Visualisierung des Interpretationsvorgangs als Baumstruktur erreicht. Der integrierte Editor ermöglicht weiterhin das Schreiben von Prolog-Quelltext. Abfragen zur Interpretation können über ein Textfeld gestellt werden, Informationen an den Nutzer wie das Finden einer Lösung werden über ein Ausgabefeld dargestellt.

Die Entwicklung von VIPER erfolgt in der Programmiersprache Java anhand des Model-View-Controller-Architekturstils. Hiernach wird die grafische Darstellungs-Komponente (View) von internen Datenstrukturen und deren Logik (Model) getrennt. Die Vermittlung zwischen diesen Komponenten erfolgt über den Controller. Die grafische View-Komponente ist hierbei für Desktop-Nutzung ausgelegt und mit Swing implementiert. Die Generierung der Visualisierung erfolgt mit Graphviz über die Graphviz-Java Implementierung, die Darstellung der erzeugten Visualisierung nutzt die Apache Batik Bibliothek.

Der folgende Bericht beschreibt den Verlauf sowie die Ergebnisse der Implementierung. Zu Beginn folgt eine Beschreibung der Erfüllungsrate der Kriterien des Pflichtenhefts. Ebenfalls dokumentiert ist der zu Beginn der Phase angesetzte Implementierungsplan und die Abweichungen von diesem. Dazu kommen notwendige Rückkopplungen zum Entwurf sowie eine Übersicht der Unit-Tests und der Statistiken der Implementierung.

\section{Übersicht: Implementierte Kriterien}

\subsection{Implementierte Muss-Kriterien}
	\textbf{Erfüllungsrate: 10/10 (100\%)}
	\begin{enumerate}[label=\textbf{M\arabic*}]
		\item Erstellen eines neuen Prolog-Programms\hfill\cmark
		\item Öffnen eines Prolog-Programms\hfill\cmark
		\item Bearbeiten eines Prolog-Programms\hfill\cmark
		\item Speichern eines Prolog-Programms\hfill\cmark
		\item Parsen eines Prolog-Programms mit begrenztem Sprachumfang\hfill\cmark
		\item Eingabe von Abfragen\hfill\cmark
		\item Einzelschritt-Interpretation von Abfragen\hfill\cmark
		\item Visualisierung als Graph\hfill\cmark
		\item Ausgabe von Lösungen zu einer Abfrage\hfill\cmark
		\item Navigation in der Visualisierung\hfill\cmark
	\end{enumerate}

\subsection{Implementierte Kann-Kriterien}
	\textbf{Erfüllungsrate: 11/12 (91,67\%)}
	\begin{enumerate}[label=\textbf{K\arabic*}]
		\item Zurückschreiten\hfill\cmark
		\item Ausführung bis zur nächsten Lösung\hfill\cmark
		\item Cut\hfill\cmark
		\item Arithmetik\hfill\cmark
		\item Standardbibliothek\hfill\cmark
		\item Standardbibliothek deaktivieren und aktivieren\hfill\cmark
		\item Export der Visualisierung als LaTeX-Code\hfill\xmark
		\item Export der Visualisierung als Bilddatei\hfill\cmark
		\item \enquote{Pretty-Printing} von Listen\hfill\cmark
		\item Formatierung von Quelltext\hfill\cmark
		\item Wechsel der Sprache zwischen Deutsch und Englisch\hfill\cmark
		\item Schnellzugriffsleiste\hfill\cmark
	\end{enumerate}

\subsection{Zusätzliche Features}
	\subsubsection{Automatische Anpassung des Fenstertitels}
		Beim Öffnen einer Datei wird der absolute Pfad der geöffneten Datei im Fenstertitel angezeigt.
	
	\subsubsection{History-Funktion des Eingabefelds}
		Gestellte Abfragen im Eingabefeld werden bis zu einem erneuten Parsen gespeichert. Dies ermöglicht das \enquote{Scrollen} durch bisher gestellte Abfragen mittels der der Pfeiltasten.
		
	\subsubsection{Zoom-Funktion des Editors}
		Die Textgröße im Editor-Fenster kann über das Bewegen des Mausrads bei gedrückter Steuerungs-Taste, oder durch Drücken der Plus und Minus Tasten in Kombination mit der Steuerungs-Taste angepasst werden.
\section{Implementierungsplan}
\begin{ganttchart}[x unit = 1.4cm, y unit chart = 0.95cm]{0}{4}
	\gantttitle{Implementierungsplan (Schritte in Wochen)}{5}\\
	\gantttitlelist{0,..., 4}{1}\\
	%%%%%%%% Woche 1
	\ganttbar{Setup}{0}{0}\\
	\ganttbar{Kommando-Teilmenge A}{1}{1}\\
	\ganttbar{GUI-Grundgerüst}{1}{1}\\
	\ganttbar{Terme und Subklassen}{1}{1}\\

	%%%%%%%% Woche 2 / 3
	\ganttbar{Kommando-Teilmenge B}{2}{2}\\
	\ganttbar{Konsole und Visualisierungsbereich}{2}{2}\\
	\ganttbar{Parser, KnowledgeBase, Ziele und Regeln}{2}{2}\\
	\ganttbar{ActivationRecords, Visitors und Unifikation}{2}{2}\\
	\ganttbar{Manager-Klassen}{2}{2}\\
	\ganttbar{Controller-Tests}{2}{3}\\

	%%%%%%%% Woche 3
	\ganttbar{Visualisierung}{3}{3}\\
	\ganttbar{Kann-Kriterien}{3}{3}\\

	%%%%%%%% Woche 4
	\ganttbar{Puffer}{4}{4}\\
\end{ganttchart}
\newpage
Der Implementierungs-Arbeit teilt sich in wöchentliche Aufgabenbereiche ein. Hierbei werden die Komponenten der Software soweit wie möglich separat entwickelt, ein Mergen in den Master-Branch findet halbwöchentlich statt. Sollte ein Merge dringend nötig sein, weil Abhängigkeiten zu anderen Code-Komponenten bestehen, so können auch Merges abseits des Zeitplans auftreten. Die Unterteilung der Aufgabenbereiche orientiert sich grob an der Paketstruktur und unterteilt diese ggf. bei größeren Aufgabenfeldern wie dem Interpreter-Manager. Dokumentation in Form von JavaDoc wird parallel zur Entwicklung geschrieben. Dies gilt analog für die Tests des Models, da dieses frühzeitig getestet werden muss. Aufgrund der Abhängigkeiten zum Rest der Software werden Tests für die View- und Controller-Komponente nicht parallel, sondern im Nachhinein entwickelt.

Das Gantt-Diagramm entspricht dem Zeitplan in Wocheneinheiten. Woche 0 entspricht hierbei dem Vorgehen am ersten Tag der ersten Woche. Die zugehörigen Aufgaben setzen sich wie folgt zusammen:
\begin{enumerate}
	\item \textbf{Setup}: Generierung des Code-Gerüsts aus dem UML-Klassendiagramm des Entwurfs, Aufsetzen des Build-Systems (Gradle) sowie des Code-Stil-Prüfers (Checkstyle). Dieser Punkt baut eine gemeinsame Grundlage auf, von der aus die einzelnen Aufgabenbereiche in separaten Branches weiterarbeiten können.
	
	\item \textbf{Kommando-Teilmenge A}: Implementierung simpler Kommandos mit wenigen bis keinen Abhängigkeiten zum Rest des Codes. Dies umfasst in erster Linie Eingabe-/Ausgabe-Kommandos sowie Kommandos, welche an den Interpreter-Manager und Sprachen-Manager des Controller-Pakets delegieren, namentlich:
	\begin{multicols}{3}
		\begin{itemize}
			\item CommandNew
			\item CommandOpen
			\item CommandSave
			\item CommandZoom

			\item CommandToggleLib
			\item CommandSetLang
			\item CommandExportTikZ
			\item CommandExportImage

			\item CommandParse
			\item CommandExit
		\end{itemize}
	\end{multicols}

	\item \textbf{GUI-Grundgerüst}: Implementierung der wichtigsten GUI-Komponenten. Hierzu gehört das Hauptfenster, der Editor sowie Menü- und Werkzeugleiste.
	
	\item \textbf{Terme und Subklassen}: Implementierung und Testen von der Term-Klasse und ihren Subklassen.
	
	\item \textbf{Abschluss der ersten Woche}: Das Projekt ist aufgesetzt. Beim Ausführen wird die GUI erfolgreich angezeigt, ein Editor ist vorhanden, Texteingaben sind möglich. Simple Dateioperationen wie Öffnen und Speichern funktionieren. Der grundlegende Term-Baustein des Models ist implementiert und getestet.

	\item \textbf{Kommando-Teilmenge B}: Implementierung komplexerer Kommandos mit diversen Abhängigkeiten zum Rest des Codes. Dies umfasst in erster Linie Kommandos, welche die Interpretation steuern. Weiter werden Kommandos fertig implementiert, die stärkere Abhängigkeiten zum Rest des Codes haben wie die Export-Funktionen:
	\begin{multicols}{3}
		\begin{itemize}
			\item CommandNextStep
			\item CommandPreviousStep
			\item CommandContinue
			\item CommandCancel
			
			\item CommandExportTikZ
			\item CommandExportImage
			\item CommandFormat
		\end{itemize}
	\end{multicols}

	\item \textbf{Konsole und Visualisierungsbereich}: Implementierung des Konsolen-Bereichs, des Visualisierungs-Bereichs sowie erster Tests. Weiter werden erste Icons für die Schaltflächen erstellt.

	\item \textbf{Parser, KnowledgeBase, Ziele und Regeln}: Implementierung der KnowledgeBase, Ziele und Regeln sowie Einfügen und Anpassen des gegebenen Parsers. Hierzu gehört ebenfalls das parallele Entwickeln von Unit-Tests.
	
	\item \textbf{ActivationRecords, Visitors und Unifikation}: Implementierung des ActivationRecords sowie des spezifischen FunctorActivationRecords, der Visitor-Klassen sowie der Unifikations-Klassen. Hierzu gehört ebenfalls das parallele Entwickeln von Unit-Tests.
	
	\item \textbf{Manager-Klassen}: Implementierung der Manager-Klassen des Controllers, namentlich dem Sprachen-Manager und dem Interpreter-Manager.
	
	\item \textbf{Controller-Tests}: Implementierung der Tests für das Controller-Paket. Dies beginnt in der zweiten Woche parallel zur Fertigstellung des Controllers und endet mit der dritten Woche.
	
	\item \textbf{Abschluss der zweiten Woche}: Die Controller-Komponente ist vollständig implementiert und zu Teilen getestet. Die grafische Benutzeroberfläche ist ebenfalls vollständig und an die Kommandos angebunden. Das Parsen von Quellcode und das Parsen sowie Interpretieren von simplen Abfragen (ohne Nutzung von Features der Kann-Kriterien) ist möglich.
	
	\item \textbf{Visualisierung}: Implementierung und Testen der Visualisierungs-Klassen des Models.
	\newpage
	\item \textbf{Kann-Kriterien}: Implementierung und Testen der im Pflichtenheft beschriebenen Kann-Kriterien, dazu gehören:
	\begin{multicols}{2}
		\begin{itemize}
			\item Zurückschreiten
			\item Arithmetik
			\item Cut
			\item Ausführung bis zur nächsten Lösung
			\item Unterstützung einer Standardbibliothek
			\item Export als TikZ-Code für LaTeX
			\item Export als Bild-Datei (PNG, SVG)
			\item Pretty-Printing für Listen
		\end{itemize}
	\end{multicols}
	
	\item \textbf{Abschluss der dritten Woche}: Die Controller-Komponente ist vollständig getestet. Die Visualisierung bei der Interpretation ist verfügbar und wird korrekt angezeigt. Die Kann-Kriterien des Pflichtenhefts wurden erfolgreich implementiert und getestet.
	
	\item \textbf{Puffer}: Puffer-Woche für finales Testen und Reparatur von entstandenen Bugs.
\end{enumerate}

\section{Implementierungsverlauf und Plan-Abweichungen}

\subsection{Abschluss erste Woche}
Da der Implementierungsplan im Lauf der ersten Woche entstanden ist, ist der Planungsinhalt auf den Fortschritt dieser zugeschnitten. Damit sind die Punkte bzgl. der Muss-Kriterien in der ersten Woche zwangsweise erfüllt worden. Zusätzlich wurden die Kann-Kriterien für den Sprachenwechsel und die Schnellzugriffsleiste bei der Erstellung des GUI-Grundgerüsts erledigt (Anmerkung: diese zusätzlich erfüllten Kann-Kriterien wurden in dem ursprünglich eingereichten Implementierungsplan übersehen, dies wurde hier verbessert).

\begin{itemize}
	\item \textbf{Setup}\hfill\cmark
	\item \textbf{Kommando-Teilmenge A}\hfill\cmark
	\item \textbf{GUI-Grundgerüst}\hfill\cmark
	\item \textbf{Terme und Subklassen}\hfill\cmark
	\item \textbf{Kann-Kriterien}
	\begin{itemize}
		\item \textbf{Wechsel der Sprache zwischen Deutsch und Englisch}\hfill\cmark
		\item \textbf{Schnellzugriffsleiste}\hfill\cmark
	\end{itemize}
\end{itemize}

\subsection{Abschluss zweite Woche}
Alle Kriterien der zweiten Woche abseits des Interpreter-Managers wurden plangemäß erfüllt. Dieser wurde aufgrund interner Unstimmigkeiten zur Implementierung, verspäteter Verfügbarkeit der wichtigen Schnittstellen und damit zeitlichen Problemen verzögert und in der dritten Woche implementiert.

\begin{itemize}
	\item \textbf{Kommando-Teilmenge B}\hfill\cmark
	\item \textbf{Konsolen und Visualisierungsbereich}\hfill\cmark
	\item \textbf{Parser, KnowledgeBase, Ziele und Regeln}\hfill\cmark
	\item \textbf{ActivationRecords, Visitors und Unifikation}\hfill\cmark
	\item Manager-Klassen\hfill\xmark
		\begin{itemize}
			\item \textbf{Sprachen-Manager}\hfill\cmark
			\item Interpreter-Manager\hfill\xmark
		\end{itemize}
	\item \textbf{Controller-Tests}\hfill\cmark
\end{itemize}

\subsection{Abschluss dritte Woche}
Die Muss-Kriterien wurden in der dritten Woche fertiggestellt. Die Testabdeckung des Controllers durch Unit-Tests wurde in den 70\%-Bereich vorangetrieben und gilt damit als erledigt. Zusätzlich wurde eine Teilmenge der Kann-Kriterien erfolgreich implementiert, übrige Kann-Kriterien wurden in die Puffer-Woche verschoben.

\begin{itemize}
	\item \textbf{Interpreter-Manager}\hfill\cmark
	\item \textbf{Controller-Tests}\hfill\cmark
	\item \textbf{Visualisierung}\hfill\xmark
	\item Kann-Kriterien
		\begin{itemize}
			\item Zurückschreiten\hfill\xmark
			\item Ausführung bis zur nächsten Lösung\hfill\xmark
			\item \textbf{Cut}\hfill\cmark
			\item Arithmetik\hfill\xmark
			\item Standardbibliothek\hfill\xmark
			\item \textbf{Standardbibliothek aktivieren und deaktivieren}\hfill\cmark
			\item Export der Visualisierung als LaTeX-Code\hfill\xmark
			\item \textbf{Export der Visualisierung als Bilddatei}\hfill\cmark
			\item \enquote{Pretty-Printing} von Listen\hfill\xmark
			\item \textbf{Formatierung von Quelltext}\hfill\cmark
		\end{itemize}
\end{itemize}

\subsection{Abschluss vierte Woche}
Die vierte Woche diente als allgemeine Puffer-Woche zur Reduzierung von Bugs. Zusätzlich wurden Kann-Kriterien, welche in der dritten Woche nicht erfüllt wurden, in diese Puffer-Woche geschoben. Dies führte zur Erfüllung aller übrigen Kann-Kriterien abseits des Exports als TikZ-Datei. Dieses Kriterium wurde aus Zeitmangel und u großem Arbeitsaufwand verworfen. Die Abdeckung der Kann-Kriterien entspricht damit:

\begin{itemize}
	\item \textbf{Zurückschreiten}\hfill\cmark
	\item \textbf{Ausführung bis zur nächsten Lösung}\hfill\cmark
	\item \textbf{Arithmetik}\hfill\cmark
	\item \textbf{Standardbibliothek}\hfill\cmark
	\item Export der Visualisierung als LaTeX-Code\hfill\xmark
	\item \textbf{\enquote{Pretty-Printing von Listen}}\hfill\cmark
\end{itemize}

\subsection{Vergleich der Implementierungsverläufe}
\hspace*{-1cm}
\begin{minipage}{0.5\textwidth}
\begin{ganttchart}[x unit = 1.0cm, y unit chart = 0.7cm]{0}{4}
	\gantttitle{Implementierungsplan}{5}\\
\gantttitlelist{0,..., 4}{1}\\
%%%%%%%% Woche 1
\ganttbar{Setup}{0}{0}\\
\ganttbar{Kommando-Teilmenge A}{1}{1}\\
\ganttbar{GUI-Grundgerüst}{1}{1}\\
\ganttbar{Terme und Subklassen}{1}{1}\\

%%%%%%%% Woche 2 / 3
\ganttbar{Kommando-Teilmenge B}{2}{2}\\
\ganttbar{Konsole und Visualisierungsbereich}{2}{2}\\
\ganttbar{Parser, KnowledgeBase, Ziele und Regeln}{2}{2}\\
\ganttbar{ActivationRecords, Visitors und Unifikation}{2}{2}\\
\ganttbar{Manager-Klassen}{2}{2}\\
\ganttbar{Controller-Tests}{2}{3}\\

%%%%%%%% Woche 3
\ganttbar{Visualisierung}{3}{3}\\
\ganttbar{Kann-Kriterien}{3}{3}\\

%%%%%%%% Woche 4
\ganttbar{Puffer}{4}{4}\\	
\end{ganttchart}
\end{minipage}%
\hspace*{6cm}
\begin{minipage}{0.5\textwidth}
\begin{ganttchart}[x unit = 1.0cm, y unit chart = 0.7cm]{0}{4}
	\gantttitle{Tatsächlicher Verlauf}{5}\\
\gantttitlelist{0,..., 4}{1}\\
%%%%%%%% Woche 1
\ganttbar{}{0}{0}\\
\ganttbar{}{1}{1}\\
\ganttbar{}{1}{1}\\
\ganttbar{}{1}{1}\\

%%%%%%%% Woche 2 / 3
\ganttbar{}{2}{2}\\
\ganttbar{}{2}{2}\\
\ganttbar{}{2}{2}\\
\ganttbar{}{2}{2}\\
\ganttbar{}{2}{3}\\
\ganttbar{}{2}{3}\\

%%%%%%%% Woche 3
\ganttbar{}{3}{3}\\
\ganttbar{}{3}{4}\\

%%%%%%%% Woche 4
\ganttbar{}{4}{4}\\
\end{ganttchart}
\end{minipage}

\section{Änderungen zum Entwurf}

\subsection{Paket: Model - Allgemein}

Allgemein wurde im Vergleich zum Entwurf jeder generische Template-Type von \enquote{ResultType} zu \enquote{T} bzw. \enquote{R} umbenannt, um Java-konformen Codestil zu gewährleisten.

\subsection{Paket: Model - AST}

\textbf{Neue Klassen:}\\
\begin{itemize}
	\item \textbf{ListFormatter}:\\
	Formatiert einen Term (wenn möglich) als Liste.
\end{itemize}

\textbf{Änderungen an bestehenden Klassen:}\\
\begin{itemize}
	\item \textbf{Anpassungen für BinaryOperation}:\\
	Die BinaryOperation-Klasse stellt nun Getter-Methoden für die beiden Seiten der Operation zur Verfügung (\textbf{getLhs()} und \textbf{getRhs()}). Weiterhin erhält die evaluate()-Methode keine Parameter mehr.
	\item \textbf{Abfragen mit mehreren Regeln:}\\
	Zur Implementierung von Abfragen mit mehreren Regeln hat die KnowledgeBase fortan eine Methode \textbf{withRule(Rule)}.
	\item \textbf{Anpassungen für Goal und Subklassen}:\\
	Die Goal-Klasse hat fortan eine Getter-Methode für die enthaltenen Variablen (\textbf{get-Variables()}). Der Konstruktor des spezifischen \textbf{ComparisonGoal} erhält nun zusätzlich zu zwei Termen nun zwei Strings, um das Symbol des Vergleichs zu setzen. Dies ist für die Ausgabe der Konsole und als HTML für Graphviz notwendig. Die Getter-Methoden für diese Symbol-Strings wurden ebenfalls hinzugefügt (\textbf{getSymbol()} und \textbf{getHtmlSymbol()}).
	\item \textbf{Konstruktor der Variable-Klasse}\\
	Die Variable-Klasse besitzt nun einen Konstruktor der Form \textbf{Variable(String)} mit einer textuellen Repräsentation der Variable als Parameter.
	\item \textbf{Statische atom(String)-Methode in Functor}\\
	Die Functor-Klasse besitzt fortan eine statische \textbf{atom(String)}-Methode zur Erzeugung eines Atoms (parameterloser Funktor). Diese dient lediglich der einfacheren Konstruktion.
	\item \textbf{equals(Object)-Methode für alle AST-Klassen}\\
	Die equals(Object)-Methode wurde in allen AST-Klassen implementiert um das Testen zu vereinfachen.
\end{itemize}

\subsection{Paket: Model - Interpreter}

\textbf{Neue Klassen:}\\
\begin{itemize}
	\item \textbf{Unifikationsergebnisse als eigene Klassen}\\
	Die UnificationResult-Klasse wurde zu eigenen Klassen der Form \textbf{\{Success|Fail| Error\}UnificationResult} spezialisiert. UnificationResult besitzt nun eine \textbf{error(String)}"=Fabrikmethode für ErrorUnificationResult-Objekte.
\end{itemize}

\textbf{Änderungen an bestehenden Klassen:}\\
\begin{itemize}
	\item \textbf{Methoden des Interpreters}:\\
	Die \textbf{step()}-Methode der Interpreter-Klasse gibt fortan ein StepResult zurück. Weiter wurde der Rückgabewert in \textbf{getCurrent()} und \textbf{getNext()} von \textbf{ActivationRecord} zu \textbf{Optional<ActivationRecord>} geändert, da dieser null sein kann. Die Methode \textbf{getResult()} ist aufgrund des Rückgabetyps der step()-Methode hinfällig und wurde gelöscht, ebenso wurde die Methoden \textbf{copy()} gelöscht. Die \textbf{copy()}-Methode ist nicht mehr notwendig, da fortan die Visualisierung selbst kopiert wird. Die \textbf{getAndIncrementUnificationIndex()}-Methode wurde zu \textbf{getUnificationIndex()} und \textbf{incrementUnificationIndex()} aufgespalten, damit diese Funktionalitäten unabhängig voneinander sind. Weiter wurde eine \textbf{getKnowledgeBase()}-Methode hinzugefügt.
	\item \textbf{Anpassungen an der Unifikation}:\\
	Zur Erzeugung und automatischen Prüfung auf Korrektheit existiert nun in \textbf{Unifier} die Methode \textbf{createSubstitutionResult(Variable, Term)}. Weiter wurde eine abstrakte Getter-Methode für den Term der Form \textbf{getTerm()} eingefügt. Weiter muss nun nicht mehr jede Subklasse von Unifier alle visit-Methoden überschreiben, stattdessen wird eine einzelne visit-Methode der Form \textbf{visit(\{Number|Variable|Functor\})} angeboten.
	\item \textbf{Enumerationselement in StepResult}:\\
	StepResult besitzt nun das Element \textbf{FROM\_STEPBACK} für die Zurückschreiten-Funktionalität.
	\item \textbf{Änderung der Oberklasse von SubstitutionApplier}:\\
	Die Oberklasse von \textbf{SubstitutionApplier} ist fortan kein TermTransformationVisitor mehr sondern ein allgemeinerer \textbf{TermVisitor}.
	\item \textbf{Neue Methode des Environments}:\\
	Die Environment-Klasse besitzt nun eine Getter-Methode für die enthaltenen Substitutionen namens \textbf{getSubstitutions()}.
	\item \textbf{Methoden der ActivationRecords}:\\
	Die \textbf{getPrevious()}-Methode gibt nun einen \textbf{optionalen} ActivationRecord zurück, da dieser null sein kann. Weiter wurde eine Methode zur Anwendung bisheriger Substitutionen eingefügt (\textbf{applyPreviousSubstitutions(Term)}), sowie Getter-Methoden für das Ziel (\textbf{getGoal()}), den ActivationRecord, der in einem Baum am weitesten rechts liegt (\textbf{getRightmost()}) sowie für einen boolschen Wert, ob der ActivationRecord erfüllt wurde (\textbf{isFulfilled()}). Die \textbf{step()}-Methode des ActivationRecords gibt nun auch einen \textbf{optionalen} ActivationRecord zurück. Die Methode \textbf{hasBeenVisited()} wurde weiterhin umbenannt zu \textbf{isVisited()}.
	\item \textbf{Geänderte Methodennamen spezifischer ActivationRecords}:\\
	Die \textbf{getSubSteps()}-Methode des FunctorActivationRecords wurde umbenannt zu \textbf{getChildren()}. Die \textbf{getUnificationResult()}-Methode des UnificationActivationRecords und des ArithmetikUnificationRecords wurde umbenannt zu \textbf{getResult()}. Die \textbf{getEvaluationError()}-Methode des ComparisonActivationRecords wurde umbenannt zu \textbf{getErrorMessage()}.
	\item \textbf{Geänderte Methoden des ArithmeticActivationRecords}:\\
	Die \textbf{getEvaluationError()}-Methode in ArithmeticActivationRecord wird nicht benötigt und wurde gelöscht. Um die arithmetisch ausgewertete rechte Seite auslesen zu können wurde die \textbf{getEvaluatedRhs()}-Methode eingefügt.
\end{itemize}

\subsection{Paket: Model - Parser}

\textbf{Neue Klassen:}\\
Keine.

\textbf{Änderungen an bestehenden Klassen:}\\
Keine (die Anpassung an die eigenen Datenstrukturen des AST sind keine Änderung zum Entwurf, da diese bereits in letzterem dokumentiert sind).

\subsection{Paket: Model - Visualisation}

\textbf{Neue Klassen:}\\
Keine.

\textbf{Änderungen an bestehenden Klassen:}\\
Die Klasse LatexMaker wurde gelöscht, da das Kann-Kriterium für den LaTeX-Export verworfen wurde.

\subsection{Paket: View}

\textbf{Neue Klassen:}\\
\begin{itemize}
	\item \textbf{Subklassen von Swing-Elementen}
	Es wurden eigene Implementierungen von Swing-Elementen angefertigt, darunter \textbf{Menu} (erbt von JMenu), \textbf{ToolBarButton} (erbt von JButton) und \textbf{CheckBoxMenuItem} (erbt von JCheckBoxMenuItem). Diese dienen der Anbindung von Swing-Elementen an die intern genutzten Entwurfsentscheidungen (Observer für Sprachen-Manager, Umschalten des ClickableState).
	\item \textbf{LanguageKey}:\\
	Sammelklasse für valide Sprachschlüssel. Diese werden nun als zentrale Enumeration verwaltet, um das Übergeben invalider Sprachschlüssel an den Sprachen-Manager zu verhindern.
	\item \textbf{ColorScheme}:\\
	Sammelklasse für die genutzten Farbwerte (vor allem für die Ausgabe im Konsolenbereich). Diese werden nun als zentrale Enumeration verwaltet, um die Farbwahl zu vereinheitlichen.
	\item \textbf{LogType}:\\
	Sammelklasse für Ausgabetypen im Ausgabebereich der Konsole. Diese werden als zentrale Enumeration verwaltet, um die Wahl über die farbliche Darstellung der Ausgabe der Konsole zu überlassen.
	\item \textbf{ClickableState} und \textbf{HasClickable}:\\
	Alle Elemente der grafischen Oberfläche, welche Schaltflächen besitzen, implementieren fortan das Interface \textbf{HasClickable} und haben damit die Methode \textbf{switchClickableState(ClickableState state)}. Der \textbf{ClickableState}-Parameter entspricht einer Enumeration aller möglichen Zustände, die Einfluss auf die Aktivierung/Deaktivierung der Schaltflächen nehmen. Beim Aufruf der Methode werden somit anhand des ClickableState alle Schaltflächen des GUI-Elements jeweils aktiviert oder deaktiviert. Kommando-Klassen erhalten eine zentrale Methode des Hauptfensters, welches den ClickableState für alle Unterelemente die HasClickable implementieren aktualisiert. Dies ermöglicht bspw. die Deaktivierung der Schaltfläche für den nächsten Schritt einer Interpretation bevor eine Abfrage gestellt wurde. Diese Schaltfläche wird nach dem erfolgreichen Stellen einer Abfrage dann automatisch aktiviert und kann betätigt werden.
\end{itemize}

\textbf{Änderungen an bestehenden Klassen:}\\
Buttons besitzen nun Tooltips, welche im Konstruktor als String übergeben werden. Das ConsoleInputField ist nun nicht mehr nur ein selbst implementiertes JTextField, sondern ein JPanel, welches ein JTextField enthält. Dies wurde geändert, um einen Button zur Verarbeitung der eingegebenen Abfrage und ein Text-Label mit einem Fragezeichen an das Eingabefeld anzuhängen. Zusätzlich hat das Eingabefeld nun eine History-Funktion und speichert bereits eingegebene Abfragen bis zum Schließen des Programms.

Die printLine-Methode des Konsolenbereichs erhält nun keine beliebige Java-Color als Parameter, sondern delegiert die Farbwahl an die Konsole und erhält stattdessen eine LogType-Enumeration. Das EditorPanel erlaubt nun die Skalierung des Texts über die Tastenkombination STRG und \enquote{+} bzw. \enquote{-}. Zuletzt wird nun auch automatisch beim Ändern der aktuell geöffneten Datei der Fenstertitel angepasst, um den absoluten Pfad der Datei anzuzeigen.

\subsection{Paket: Controller}
\textbf{Neue Klassen:}\\
\begin{itemize}
	\item \textbf{CommandExit}:\\
	Schließt das Programm und fragt bei ungesicherten Änderungen, ob diese gespeichert werden sollen. Dieses Kommando wurde für bessere Konsistenz sowie das Menü-Element \enquote{Schließen} hinzugefügt.
	\item \textbf{CommandParseQuery}:\\
	Verarbeitet eine gestellte Abfrage durch den Parser und schaltet bei Erfolg die Interpretation frei. Dies ist zu trennen von CommandParse, welches sich nun auf den Quellcode im Editor beschränkt. Besagte Trennung vereinfacht das Parsen-Kommando, welches nun keine interne Fallunterscheidung mehr durchführen muss.
	\item \textbf{FileFilters}:\\
	Sammelklasse für oft genutzte, konstante, statische Dateifilter, welche in Kommandos mit Dateidialogen genutzt werden.
	\item \textbf{FileUtilities}:\\
	Sammelklasse für oft genutzte statische Dateioperationen, welche in Kommandos mit Dateidialogen genutzt werden. Dies umfasst das Anfügen einer Dateiendung, falls nötig.
	\item \textbf{LanguageKey}:\\
	Enumeration aller möglichen Schlüsselwerte, über die eine Übersetzung durch den LanguageManager angefordert werden kann. Dies verhindert die Übergabe von nicht definierten Schlüsselwerten.
	\item \textbf{PreferencesManager}:\\
	Manager-Klasse für das Speichern und Laden von Produktdaten.
\end{itemize}

\textbf{Änderungen an bestehenden Klassen:}\\
Die öffentlichen execute()-Methoden der Kommandos teilen die Ausführungsschritte in intern definierte Funktionen zur Vereinfachung des Programmcodes. Diese sind ausschließlich zu Testzwecken als \texttt{public} deklariert und werden dementsprechend nicht zusätzlich dokumentiert. Aufgrund von Schnittstellen-Anpassungen haben sich die Konstruktoren in diversen Kommandos des Controllers verändert. Die meist vertretenen Anpassungen umfassen hierbei die Übergabe von Consumer-Funktionen, über die einzelne isolierte Funktionen des Hauptfensters ausgeführt werden können.

Dies umfasst einerseits die Consumer-Funktion für das Ändern des Fenstertitels, welche nun an Kommandos übergeben wird, welche die aktuell geöffnete Datei beeinflussen:
\begin{multicols}{3}
	\begin{itemize}
		\item CommandNew
		\item CommandOpen
		\item CommandSave
	\end{itemize}
\end{multicols}

Andererseits existiert eine Consumer-Funktion für das globale Ändern des ClickableStates, welche nun an Kommandos übergeben wird, welche diesen beeinflussen:
\begin{multicols}{3}
	\begin{itemize}
		\item CommandNew
		\item CommandOpen
		\item CommandParse
		\item CommandNextStep
		\item CommandToggleLib
	\end{itemize}
\end{multicols}

Zusätzlich wird der InterpreterManager nun häufiger übergeben, da einige Commands den \enquote{Nächste Lösung}-Thread beenden und auf dessen tatsächliche Terminierung warten müssen. Durch die Delegation des Threadings für das \enquote{Nächste Lösung}-Kommando und das \enquote{Abbrechen}-Kommando an den InterpreterManager fallen auch die boolschen running-Parameter für diese weg. Diese Änderung betrifft:
\begin{multicols}{3}
	\begin{itemize}
		\item CommandNew
		\item CommandOpen
		\item CommandSave
		\item CommandContinue
		\item CommandCancel
	\end{itemize}
\end{multicols}

Weiter wurde das CommandContinue zu CommandNextSolution umbenannt. Zusätzlich erhält das CommandPreviousStep nicht mehr das ConsolePanel, da dieses keine Ausgabe machen muss. Da das VisualisationPanel keinen Getter für die Visualisierung mehr besitzt und diese stattdessen selbstständig per GraphvizMaker erstellt wird, wurde dieses als Parameter in den Export-Kommandos entfernt. Der hinzugefügte PreferenceManager ist nun weiterhin ein Parameter von CommandToggleLib und CommandSetLang, da diese Produktdaten ändern welche gesichert werden müssen. Zuletzt erhält CommandOpen ein Speichern-Kommando per Konstruktor, sodass dieses nicht dupliziert werden muss.

\section{Unit-Tests}

\subsection{Übersicht}
\begin{figure}[!h]
	\centering
	\begin{tabular}{l | c | c | c | c}
		\hline
		Paket:			& Testzahl & Erfüllt & Fehlschlagend & Ignoriert \\
		\hline
		Gesamt:			& 84 	& 79 & 0 & 5\\
		\hline
		Controller: 	& 21	& 19 & 0 & 2\\
		View:			& 0		& 0 & 0 & 0\\
		Parser:			& 2		& 2 & 0 & 0\\
		AST:			& 37	& 34 & 0 & 3\\
		Interpreter:	& 23	& 23 & 0 & 0\\
		Visualisation:	& 1		& 1 & 0 & 0\\
		\hline
	\end{tabular}
\end{figure}

\subsubsection{Unit-Tests: Controller}
\begin{figure}[!h]
	\centering
	\begin{tabular}{l | c | c | c | c}
		\hline
		Klasse:			& Testzahl & Erfüllt & Fehlschlagend & Ignoriert \\
		\hline
		CommandExportImageTest 	& 2 & 0 & 0 & 2\\
		CommandFormatTest		& 1 & 1 & 0 & 0\\
		CommandNewTest			& 1 & 1 & 0 & 0\\
		CommandOpenTest			& 3 & 3 & 0 & 0\\
		CommandParseTest		& 3 & 3 & 0 & 0\\
		CommandSaveTest			& 4 & 4 & 0 & 0\\
		CommandSetLangTest		& 1 & 1 & 0 & 0\\
		CommandToggleLibTest	& 1 & 1 & 0 & 0\\
		FileFiltersTest			& 4 & 4 & 0 & 0\\
		FileUtilitiesTest		& 1 & 1 & 0 & 0\\
		\hline
	\end{tabular}
\end{figure}

\subsubsection{Unit-Tests: Parser}
\begin{figure}[!h]
	\centering
	\begin{tabular}{l | c | c | c | c}
		\hline
		Klasse:		& Testzahl & Erfüllt & Fehlschlagend & Ignoriert \\
		\hline
		ParserTest 	& 2 & 2 & 0 & 0\\
		\hline
	\end{tabular}
\end{figure}
\newpage

\subsubsection{Unit-Tests: AST}
\begin{figure}[!h]
	\centering
	\begin{tabular}{l | c | c | c | c}
		\hline
		Klasse:						& Testzahl & Erfüllt & Fehlschlagend & Ignoriert \\
		\hline
		AdditionOperationTest 		& 2 & 2 & 0 & 0\\		
		BinaryOperationTest 		& 5 & 4 & 0 & 1\\
		FunctorGoalTest 			& 2 & 2 & 0 & 0\\
		FunctorTest 				& 9 & 8 & 0 & 1\\		
		MultiplicationOperationTest & 2 & 2 & 0 & 0\\
		NumberTest 					& 5 & 5 & 0 & 0\\
		RuleTest 					& 3 & 3 & 0 & 0\\		
		SubtractionOperationTest 	& 2 & 2 & 0 & 0\\
		TermTest 					& 1 & 1 & 0 & 0\\
		VariableTest 				& 6 & 5 & 0 & 1\\
		\hline
	\end{tabular}
\end{figure}

\subsubsection{Unit-Tests: Interpreter}
\begin{figure}[!h]
	\centering
	\begin{tabular}{l | c | c | c | c}
		\hline
		Klasse:						& Testzahl & Erfüllt & Fehlschlagend & Ignoriert \\
		\hline
		IndexifierTest 			& 3 & 3 & 0 & 0\\		
		InterpreterTest 		& 1 & 1 & 0 & 0\\
		SubstitutionTest 		& 3 & 3 & 0 & 0\\
		UnificationResultTest 	& 7 & 7 & 0 & 0\\		
		UnificationTest 		& 8 & 8 & 0 & 0\\
		VariableExtractorTest 	& 1 & 1 & 0 & 0\\
		\hline
	\end{tabular}
\end{figure}

\subsubsection{Unit-Tests: Visualisation}
\begin{figure}[!h]
	\centering
	\begin{tabular}{l | c | c | c | c}
		\hline
		Klasse:				& Testzahl & Erfüllt & Fehlschlagend & Ignoriert \\
		\hline
		GraphvizMakerTest 	& 1 & 1 & 0 & 0\\
		\hline
	\end{tabular}
\end{figure}

\newpage
\subsection{Code-Abdeckung}
\begin{figure}[!h]
\centering
\hspace*{-1cm}
\begin{tikzpicture}[scale=1.2]
	\begin{axis}[
		xbar, 
		y=-0.5cm,
		bar width=0.3cm,
		xmin = 0,
		xmax = 100,
		enlarge y limits={abs=0.45cm},
		xlabel={Gesamtabdeckung in Prozent},
		symbolic y coords={
			Instruktions-Abdeckung,
			Branch-Abdeckung
		},
		ytick=data,
		nodes near coords, nodes near coords align={horizontal},
		]
		\addplot table[col sep=comma,header=false] {
			77, Instruktions-Abdeckung
			60, Branch-Abdeckung
		};
	\end{axis}
\end{tikzpicture}\\\vspace*{1cm}\hspace*{1cm}
\begin{tikzpicture}[scale=1.2]
	\begin{axis}[
		xbar, 
		y=-0.5cm,
		bar width=0.3cm,
		xmin = 0,
		enlarge y limits={abs=0.45cm},
		xlabel={Abgedeckte Instruktionen in Prozent},
		symbolic y coords={
			Controller,
			View,
			Parser,
			AST,
			Interpreter,
			Visualisation
	    },
		ytick=data,
		nodes near coords, nodes near coords align={horizontal},
		]
		\addplot table[col sep=comma,header=false] {
			65, Controller
			77, View
			70, Parser
			92, AST
			95, Interpreter
			98, Visualisation
		};
	\end{axis}
\end{tikzpicture}\\\vspace*{1cm}\hspace{1cm}
\begin{tikzpicture}[scale=1.2]
	\begin{axis}[
		xbar, 
		y=-0.5cm,
		bar width=0.3cm,
		xmin = 0,
		enlarge y limits={abs=0.45cm},
		xlabel={Abgedeckte Branches in Prozent},
		symbolic y coords={
			Controller,
			View,
			Parser,
			AST,
			Interpreter,
			Visualisation,
		},
		ytick=data,
		nodes near coords, nodes near coords align={horizontal},
		]
		\addplot table[col sep=comma,header=false] {
			33, Controller
			24, View
			59, Parser
			75, AST
			92, Interpreter
			83, Visualisation
		};
		\end{axis}
	\end{tikzpicture}
\end{figure}


\section{Statistiken}

% Commits, LoC, JavaDoc, ...
\subsection{Commits}
\begin{figure}[!h]
	\centering
	%% Merge-Commits sind abgezählt, Rest via git rev-list <HASH> --count
	Verlauf der Commit-Zahlen\\\vspace*{0.3cm}
	\begin{tabular}{l | c | c | c | c}
		\hline
		Datum des Abrufs:	& 2. Juli 2018 & 9. Juli 2018 & 16. Juli 2018 & 23. Juli 2018\\
		\hline
		Differenz zur vorigen Woche:	& 64 & 130 & TODO & TODO \\
		Differenz zum Phasenbeginn: & 64 & 194 & TODO & TODO \\
		Merge-Commits in Woche: & 10 & 14 & 16 & TODO \\
		\hline
	\end{tabular}
\end{figure}
\vspace*{1cm}
\begin{minipage}{0.5\textwidth}
		\centering
		Commits pro Woche:\\
		\begin{tikzpicture}[scale=0.9]
		\begin{axis}[
		symbolic x coords = {1. Woche, 2. Woche, 3. Woche, 4. Woche}, xtick = data]
		
		\addplot[ybar, fill = blue] coordinates {
			(1. Woche, 64)
			(2. Woche, 194)
			(3. Woche, 0)
			(4. Woche, 0)
		};
		\end{axis}
		\end{tikzpicture}
\end{minipage}\hspace*{0.7cm}
\begin{minipage}{0.5\textwidth}
		\centering
		Merge-Commits pro Woche:\\
		\begin{tikzpicture}[scale=0.9]
		\begin{axis}[
		symbolic x coords = {1. Woche, 2. Woche, 3. Woche, 4. Woche}, xtick = data]
		
		\addplot[ybar, fill = blue] coordinates {
			(1. Woche, 10)
			(2. Woche, 14)
			(3. Woche, 16)
			(4. Woche, 0)
		};
		\end{axis}
		\end{tikzpicture}
\end{minipage}

\subsection{Codebase}
\begin{figure}[!h]
	\centering
	\begin{tabular}{c | c | c | c | c}
		\hline
		%%% Berechnet via CLOC
		     & Dateien & Leerzeilen & Kommentare & Code\\
		\hline
		Hauptcode & X & X & X & X\\
		\hline
		Testcode & X & X & X & X\\
		\hline
		Gesamt & 123 & 1327 & 3139 & 4761\\
	\end{tabular}
\end{figure}


\end{document}
